\documentclass{article}

\usepackage[T1]{fontenc}
\usepackage[polish]{babel}
\usepackage[utf8]{inputenc}
\usepackage{amsmath, amssymb}

\title{\normalsize Określenie regularności języka $L = \{ ww^R : w \in \{0,1\}^+, w \neq \varepsilon \}$}
\author{\normalsize Zofia Tarchalska, indeks: 279699}
\date{}

\begin{document}
\maketitle


\section{Treść zadania}
Rozważmy język


\[
L=\{\, w w^R x \mid w,x\in\{0,1\}^* \,\},
\]


gdzie \(w^R\) oznacza słowo \(w\) odwrócone. Należy ustalić, czy \(L\) jest językiem regularnym.

\section{Idea dowodu}
Twierdzenie Myhilla-Nerode’a orzeka, że język L jest regularny wtedy i tylko wtedy, gdy relacja sufiksowej nieodróżnialności generuje nam skończenie wiele klas abstrakcji.
Pokażemy, że \(L\) nie jest regularny poprzez znalezienie sufiksu, który rozróżnia dwa słowa dla każdej pary indeksów co daje nam nieskończenie wiele klas.

\section{Dowód}
Dla \(n\ge 0\) zdefiniujmy


\[
s_n=(01)^n,\qquad t_n=(10)^n1.
\]


\noindent Zauważmy najpierw, że słowa \(s_n t_n\) należą do \(L\), bo


\[
s_n t_n=(01)^n(10)^n1 = w\,w^R x
\]
dla \(w=(01)^n\) i \(x=1\).
Weźmy dwie różne liczby naturalne \(n\) i \(m\) i bez utraty ogólności załóżmy \(n<m\). Pokażemy, że sufiks \(t_n\) rozróżnia \(s_n\) i \(s_m\), tzn.


\[
s_n t_n \in L \quad\text{oraz}\quad s_m t_n \notin L.
\]
Załóżmy przeciwnie, że \(s_m t_n\in L\). Wtedy istnieją słowa \(u,v\in\{0,1\}^*\) takie, że


\[
s_m t_n = u\,u^R v.
\]
Niech \(k=|u|\). Rozważymy pozycję \(u\) względem prefiksu \(s_m\).

\subsection{Przypadek 1: \(k<|s_m|\)}
W tym przypadku część \(u\) kończy się wewnątrz naprzemiennego wzoru \(s_m=(01)^m\). Ostatni znak \(u\) musi być zgodny z naprzemiennym wzorcem \(s_m\), a pierwszy znak \(u^R\) jest równy temu ostatniemu znakowi. To prowadzi do wymogu, by w odpowiednich pozycjach, gdzie łączy się \(u\) i \(u^R\) znajdowały się dwa takie same bity pod rząd. Jednak po każdym 0 w \(s_m\) występuje 1, a po każdym 1 występuje 0. To równocześnie nie może zajść. Zatem nie może zajść \(k<|s_m|\).

\subsection{Przypadek 2: \(k\ge|s_m|\)}
Jeśli \(k\ge|s_m|\), to \(u\) zawiera cały prefiks \(s_m\). W takim wypadku \(u^R\) musi znaleźć się w pozostałej części słowa \(s_m t_n\). Jednak ponieważ \(n<m\), pozostała część \(t_n\) ma długość mniejszą niż \(k\), więc nie wystarczy miejsca na pełne \(u^R\). To również prowadzi do sprzeczności.

\noindent \\Z powyższych przypadków wynika, że założenie \(s_m t_n\in L\) jest fałszywe, stąd \(s_m t_n\notin L\).

\section{Wniosek}
Dla dowolnych dwóch różnych indeksów \(n\ne m\) istnieje sufiks (np. \(t_n\)), który rozróżnia \(s_n\) i \(s_m\). Oznacza to, że słowa \(s_n\) należą do nieskończenie wielu różnych klas równoważności Myhilla–Nerode. W konsekwencji \(L\) nie ma skończonej liczby klas równoważności i nie jest akceptowany przez żaden deterministyczny automat skończony. Zatem \(L\) nie jest językiem regularnym.

\end{document}
