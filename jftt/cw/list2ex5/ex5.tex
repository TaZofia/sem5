\documentclass{article}

\usepackage[T1]{fontenc}
\usepackage[polish]{babel}
\usepackage[utf8]{inputenc}
\usepackage{amsmath, amssymb}

\title{\normalsize Dowód nieregularności języka $L = \{ ww^R : w \in \{0,1\}^+, w \neq \varepsilon \}$}
\author{\normalsize Zofia Tarchalska}
\date{}

\begin{document}
\maketitle

\noindent Rozważamy język:
\[
L = \{ ww^R : w \in \{0,1\}^+, w \neq \varepsilon \},
\]
gdzie $w^R$ oznacza słowo $w$ zapisane w odwrotnej kolejności.

\section*{Twierdzenie}
Język $L$ nie jest regularny.

\section*{Dowód}
Załóżmy nie wprost, że $L$ jest regularny. Wtedy spełnia lemat o pompowaniu dla języków regularnych. Niech $p$ będzie długością pompowania wynikającą z lematu o pompowaniu.

\noindent Weźmy słowo:
\[
s = 0^p 1 1 0^p,
\]
które należy do $L$, ponieważ można je zapisać jako $s = w w^R$ dla $w = 0^p 1$.

\noindent Zgodnie z lematem o pompowaniu istnieje rozkład $s = xyz$ taki, że:
\[
|xy| \le p,\quad |y| > 0.
\]
Warunek $|xy| \le p$ oznacza, że zarówno $x$, jak i $y$ składają się wyłącznie z pierwszego segmentu zer. Zatem:
\[
y = 0^k \quad \text{dla pewnego } k \ge 1.
\]

\noindent Napompujmy słowo, biorąc $i = 2$. Otrzymujemy słowo:
\[
s' = xy^2z = 0^{p+k} 1 1 0^p.
\]

\noindent Jeśli $s'$ należałoby do $L$, to musiałoby mieć postać $u u^R$, co wymaga, aby liczba zer przed środkowymi jedynkami była równa liczbie zer po nich. Jednak:
\[
p + k \neq p,
\]
co prowadzi do sprzeczności.

\noindent Zatem założenie, że $L$ jest regularny, jest fałszywe.

\begin{flushright}
\(\Box\)
\end{flushright}

\end{document}
