\documentclass{article}
\usepackage{graphicx}
\usepackage[T1]{fontenc}
\usepackage[utf8]{inputenc}
\usepackage[polish]{babel}
\usepackage{amsmath}
\usepackage{booktabs}

\title{Laboratorium 2 - sprawozdanie}
\author{Zofia Tarchalska}
\date{}

\begin{document}
\maketitle

\section{Zadanie 1}
Optymalizacja dostaw paliwa

\subsection*{1. Opis modelu}

\textbf{Zmienna decyzyjna:}

\begin{itemize}
  \item $x_{s,a}$ - ilość jednostek paliwa dostarczona przez dostawcę $s$ na lotnisko $a$ [jednostki paliwa]
\end{itemize}

\noindent \textbf{Ograniczenia:}

\begin{itemize}
  \item Dostawca nie może dostarczyć więcej niż jego maksymalna dostępność:
  

\[
\sum_{a \in A_s} x_{s,a} \leq \text{supply}_s \quad \forall s \in S
\]

gdzie $S$ to zbiór wszystkich dostawców


\item Każde lotnisko musi otrzymać dokładnie tyle paliwa, ile potrzebuje:
  

\[
\sum_{s \in S_a} x_{s,a} = \text{demand}_a \quad \forall a \in A
\]

gdzie $A$ to zbiór wszystkich lotnisk

\item Dostawy są możliwe tylko dla dozwolonych par $(s,a)$.
\end{itemize}

\noindent \textbf{Funkcja celu:}

\[
\min \sum_{(s,a) \in D} c_{s,a} \cdot x_{s,a}
\]

gdzie:
\begin{itemize}
  \item $c_{s,a}$ - koszt jednostkowy zakupu i dostawy paliwa od dostawcy $s$ na lotnisko $a$ [zł/jednostkę]
  \item $D$ - zbiór dozwolonych par dostawca - lotnisko
\end{itemize}

\subsection*{2. Opis danych i wyników}

Dane wejściowe zostały wczytane z pliku \texttt{data\_ex1.json}. Uwzględniono:
\begin{itemize}
  \item 3 dostawców paliwa o dostępności: 275000, 550000, 660000 jednostek
  \item 4 lotniska o zapotrzebowaniu: 110000, 220000, 330000, 440000 jednostek
  \item Koszty dostawy zależne od pary dostawca - lotnisko
\end{itemize}

\noindent Model został rozwiązany za pomocą solvera GLPK. Uzyskano wyniki:
\begin{verbatim}
Min cost: 8.525e6

Company F1, limit = 275000
Supplier F1 → Airport L1: 0.0
Supplier F1 → Airport L2: 165000.0
Supplier F1 → Airport L3: 0.0
Supplier F1 → Airport L4: 110000.0

F1 delivered 275000.0 in total.

Company F2, limit = 550000
Supplier F2 → Airport L1: 110000.0
Supplier F2 → Airport L2: 55000.0
Supplier F2 → Airport L3: 0.0
Supplier F2 → Airport L4: 0.0

F2 delivered 165000.0 in total.

Company F3, limit = 660000
Supplier F3 → Airport L1: 0.0
Supplier F3 → Airport L2: 0.0
Supplier F3 → Airport L3: 330000.0
Supplier F3 → Airport L4: 330000.0

F3 delivered 660000.0 in total.
\end{verbatim}

\subsection*{Wnioski}
Okazuje się że minimalny koszt wynosi $8 525 000$. Optymalny plan dostaw wskazuje, ile jednostek paliwa należy zakupić od każdego dostawcy i dostarczyć na każde lotnisko, aby spełnić zapotrzebowanie przy minimalnym koszcie. Wyniki są zgodne z ograniczeniami dostępności. Zauważmy, że nie wszytskie firmy muszą dostarczyć paliwo, aby otrzymać optymalne rozwiązanie. Jednak aż dwie firmy F1 i F3 wyczerpały możliwości dostaw (wysłały wszystko co mogły).
\newpage
\section{Zadanie 2}
Optymalizacja produkcji w fabryce
\subsection*{1. Opis modelu}

\begin{itemize}
\item \textbf{Zmienna decyzyjna:}

$x_p$ – Liczba kilogramów produktu $p \in P$ wyprodukowana w danym tygodniu [kg]

\item \textbf{Ograniczenia:}

\begin{itemize}
\item Ograniczenia czasowe dla każdej maszyny $m \in M$:

\[
\sum_{p \in P} t_{p,m} \cdot x_p \leq T_m \quad \forall m \in M
\]

gdzie:
\begin{itemize}
  \item $t_{p,m}$ – liczba godzin pracy maszyny $m$ potrzebna do wyprodukowania 1 kg produktu $p$
  \item $T_m$ – maksymalna liczba godzin pracy maszyny $m$ w tygodniu
\end{itemize}

\item Ograniczenia popytu dla każdego produktu $p \in P$:

\[
x_p \leq D_p \quad \forall p \in P
\]

gdzie:
\begin{itemize}
  \item $D_p$ – maksymalny tygodniowy popyt na produkt $p$
\end{itemize}
\end{itemize}

\item \textbf{Funkcja celu:}

\[
\max \sum_{p \in P} \left( r_p - c_p - \sum_{m \in M} w_m \cdot t_{p,m} \right) \cdot x_p
\]

gdzie:
\begin{itemize}
  \item $r_p$ – cena sprzedaży 1 kg produktu $p$ [zł/kg]
  \item $c_p$ – koszt materiałowy produkcji 1 kg produktu $p$ [zł/kg]
  \item $w_m$ – koszt pracy maszyny $m$ [zł/godz.]
  \item $t_{p,m}$ – liczba godzin pracy maszyny $m$ na 1 kg produktu $p$
\end{itemize}
\end{itemize}

\noindent Wynik działania solvera:\\
\begin{verbatim}
Optimal production strategy:
Product P1: 125.0 kg
Product P2: 100.0 kg
Product P3: 150.0 kg
Product P4: 500.0 kg

Max profit: 3632.5 zł
\end{verbatim}

\subsection*{Wnioski}
Model określa optymalną ilość produkcji każdego z produktów, uwzględniając ograniczenia czasowe maszyn oraz maksymalny popyt. Funkcja celu maksymalizuje zysk, który uwzględnia przychód ze sprzedaży, koszt materiałowy oraz koszt pracy maszyn. Okazuje sie, że wszystkie produkty oprócz $P1$ należy wyprodukować w liczbie równej maksymalnemu popytowi. Najwyższy możliwy zysk to 3632.5 zł.


\section{Zadanie 3}

\subsection{Opis modelu}

\subsubsection*{(a) Zmienne decyzyjne}

\begin{itemize}
    \item $x\_normal_j$ – liczba jednostek wyprodukowanych w okresie $j$ w trybie normalnym [jednostki]
    \item $x\_extra_j$ – liczba jednostek wyprodukowanych w okresie $j$ w trybie ponadwymiarowym [jednostki]
    \item $x\_stored_j$ – liczba jednostek magazynowanych po okresie $j$ [jednostki]
\end{itemize}

\subsubsection*{(b) Ograniczenia}

\begin{itemize}
\item Produkcja normalna w każdym okresie nie może przekroczyć 100 jednostek:

\[
x\_normal_j \leq 100
\]

\item Produkcja ponadwymiarowa w każdym okresie nie może przekroczyć limitu $a_j$:

\[
  x\_extra_j \leq a_j
\]

\item Liczba jednostek magazynowanych po każdym okresie nie może przekroczyć 70:
\[
  x\_stored_j \leq 70
\]
\item Bilans zapasów w każdym okresie musi spełniać:
\[
x\_normal_j + x\_extra_j + x\_stored_{j-1} = d_j + x\_stored_j
\]
gdzie $s_0 = 15$ to początkowy stan magazynu, a $d_j$ to zapotrzebowaniew danym okresie
\end{itemize}

\subsubsection*{(c) Funkcja celu}

\[
\min \sum_{j=1}^{4} \left( c_j \cdot x\_normal_j + o_j \cdot x\_extra_j + s_{cost} \cdot x\_stored_j \right)
\]

gdzie:
\begin{itemize}
    \item $c_j$ – koszt jednostkowy produkcji normalnej w okresie $j$ [zł]
    \item $o_j$ – koszt jednostkowy produkcji ponadwymiarowej w okresie $j$ [zł]
    \item $s_{cost}$ – koszt magazynowania jednej jednostki przez jeden okres [zł]
\end{itemize}

\subsection{Opis danych i interpretacja wyników}

Rozważany przypadek obejmuje 4 kolejne okresy produkcyjne. W każdym okresie firma może wyprodukować do 100 jednostek w trybie normalnym oraz dodatkowe jednostki w trybie ponadwymiarowym (zgodnie z limitem $a_j$). Popyt $d_j$ oraz koszty $c_j$, $o_j$ są znane dla każdego okresu. Firma może magazynować do 70 jednostek między okresami, przy początkowym stanie magazynu równym 15 jednostek.

\subsubsection*{Uzyskany plan produkcji}

\begin{verbatim}
Production plan: 

Period J1:
normal production = 100.0
additional_production = 15.0
stored= 0.0

Period J2:
normal production = 100.0
additional_production = 50.0
stored= 70.0

Period J3:
normal production = 100.0
additional_production = 0.0
stored= 45.0

Period J4:
normal production = 100.0
additional_production = 50.0
stored= 0.0

Min cost: 3.8425e6
\end{verbatim}

\subsection{Wnioski}

\begin{itemize}
    \item Minimalny łączny koszt produkcji i magazynowania wynosi 3\,842\,500 zł.
    \item Produkcja ponadwymiarowa została zaplanowana w okresach J1, J2 oraz J4.
    \item Magazyn osiągnął maksymalną pojemność (70 jednostek) w okresie J2.
\end{itemize}

\section{Zadanie 4}

\subsection{Opis problemu}

Dany jest skierowany graf $G = (N, A)$, gdzie $N$ to zbiór miast (wierzchołków), a $A$ to zbiór połączeń (łuków). Każda krawędź $(i,j) \in A$ posiada koszt przejazdu $c_{ij}$ oraz czas przejazdu $t_{ij}$. Celem jest znalezienie ścieżki z miasta początkowego $i^\circ$ do miasta końcowego $j^\circ$, której całkowity koszt jest minimalny, a całkowity czas przejazdu nie przekracza zadanego limitu $T$.

\subsection{Model matematyczny}

\textbf{Zmienne decyzyjne:}


\[
x_{ij} \in \{0,1\} \quad \text{dla każdej } (i,j) \in A
\]


$x_{ij} = 1$ oznacza, że krawędź $(i,j)$ należy do wybranej ścieżki.

\noindent \textbf{Funkcja celu:}


\[
\min \sum_{(i,j) \in A} c_{ij} \cdot x_{ij}
\]



\noindent \textbf{Ograniczenie czasu:}


\[
\sum_{(i,j) \in A} t_{ij} \cdot x_{ij} \leq T
\]



\noindent \textbf{Ograniczenia przepływu:}


\[
\sum_{i: (i,v) \in A} x_{iv} - \sum_{j: (v,j) \in A} x_{vj} =
\begin{cases}
-1 & \text{jeśli } v = i^\circ \\
1 & \text{jeśli } v = j^\circ \\
0 & \text{w przeciwnym razie}
\end{cases}
\]

\subsection{Parametry instancji}

\begin{itemize}
    \item Liczba miast: $|N| = 10$
    \item Miasto początkowe: $i^\circ = 1$
    \item Miasto końcowe: $j^\circ = 10$
    \item Limit czasu: $T = 15$
\end{itemize}

\subsection{Wyniki - zadanie z polecenia}

\begin{verbatim}
Optimal path found:
Total cost: 13.0
Total time: 15.0
Edges in the path:
1 → 2 (cost: 3, time: 4)
2 → 3 (cost: 2, time: 3)
3 → 5 (cost: 2, time: 2)
5 → 7 (cost: 3, time: 3)
7 → 9 (cost: 1, time: 1)
9 → 10 (cost: 2, time: 2)
\end{verbatim}

\subsection{Wyniki - mój przykład}
\begin{verbatim}
Optimal path found:
Total cost: 4.0
Total time: 15.0
Edges in the path:
1 → 2 (cost: 1, time: 5)
2 → 9 (cost: 2, time: 5)
9 → 10 (cost: 1, time: 5)
\end{verbatim}

\subsection{Wnioski}

Model skutecznie znajduje najtańszą ścieżkę spełniającą ograniczenie czasowe.

\noindent \\Ograniczenie na całkowitoliczbowość zmiennej decyzyjnej w tym przykładzie jest koniecze. W moim problemie zmienną decyzyjną jest zmienna binarna, w zależności czy dołączamy krawędź do rozwiązania czy nie, przyjmuje wartości 1 lub 0. Dopuszczenie zmiennych niecałkowitych byłoby kompletnie sprzeczne z logiką programu. Mogłyby nam wyjść wartości ułamkowe a nie da się tego interpretować w przypadku ścieżek w grafie lub jako watość true/false.

\noindent \\Jeśli chodzi o przypadek, w którym usunęlibyśmy ograniczenie na całkowitoliczbowość zmiennej decyzyjnej oraz na czasy przejazdu to właśnie ze względu na brak ograniczenia całkowitoliczbowości nie otrzymamy zawsze akceptowalnego rozwiązania.

\section{Zadanie 5}
Optymalny przydział radiowozów

\subsection{ Opis modelu}

\begin{itemize}
    \item $x_{ij}$ – liczba radiowozów przypisana do dzielnicy $i$ podczas zmiany $j$ [szt.]
\end{itemize}

\subsubsection*{Ograniczenia}

\begin{itemize}
  \item Dla każdej dzielnicy $i$ i zmiany $j$:
      
  \[
    \text{min}_{ij} \leq x_{ij} \leq \text{max}_{ij}
  \]


  \item Dla każdej zmiany $j$:
      
  \[
    \sum_{i} x_{ij} \geq r_j
  \]

  \item Dla każdej dzielnicy $i$:

  \[
    \sum_{j} x_{ij} \geq d_i
  \]

\end{itemize}

\subsubsection*{Funkcja celu}

\[
\min \sum_{i,j} x_{ij}
\]

\noindent czyli minimalizacja całkowitej liczby radiowozów.

\vspace{1em}
\subsection{Wyniki i interpretacja}

\begin{verbatim}
District p1
Shift s1: 2.0
Shift s2: 7.0
Shift s3: 5.0

District p2
Shift s1: 3.0
Shift s2: 6.0
Shift s3: 7.0

District p3
Shift s1: 5.0
Shift s2: 7.0
Shift s3: 6.0

Min number of vehicles: 48.0
\end{verbatim}


\subsection{Wnioski}
\begin{itemize}
    \item Łączna liczba radiowozów: 48
    \item Wszystkie wymagania dotyczące minimalnych i maksymalnych przydziałów zostały spełnione.
    \item Przydział jest efektywny i zgodny z przepisami dla każdej zmiany i dzielnicy.
\end{itemize}


\section{Zadanie 6}
Optymalne rozmieszczenie kamer

\subsection{Opis problemu}

Firma przeładunkowa składuje kontenery z cennym ładunkiem na siatce o wymiarach $m \times n$. Każdy kontener zajmuje dokładnie jeden kwadrat. Celem jest rozmieszczenie kamer w pustych kwadratach tak, aby każdy kontener był monitorowany przez co najmniej jedną kamerę, a liczba użytych kamer była minimalna.

\noindent Zasięg kamery obejmuje $k$ pól w górę, dół, lewo i prawo (czyli w formie krzyża). Kamera nie może być umieszczona w polu zajętym przez kontener.

\subsection{Zmienne decyzyjne}

\begin{itemize}
    \item $c_{ij} \in \{0,1\}$ – czy kamera znajduje się w komórce $(i,j)$
    \item $K$ – zbiór współrzędnych kontenerów
\end{itemize}

\subsection{Ograniczenia}

\begin{itemize}
  \item Kamera nie może być umieszczona na kontenerze:
    
  \[
    c_{ij} = 0 \quad \forall (i,j) \in K
  \]

 \item Każdy kontener musi być monitorowany przez co najmniej jedną kamerę w zasięgu:  

  \[
    \sum_{(p,q) \in \text{krzyż}(i,j)} c_{pq} \geq 1 \quad \forall (i,j) \in K
  \]

    gdzie $\text{krzyż}(i,j)$ to zbiór pól w zasięgu $k$ od kontenera $(i,j)$ w pionie i poziomie.
\end{itemize}

\subsection{Funkcja celu}

\[
\min \sum_{i=1}^{m} \sum_{j=1}^{n} c_{ij}
\]


\noindent czyli minimalizacja liczby użytych kamer.
\subsection{Wyniki dla siatki $6 \times 6$}
\begin{verbatim}
X - represents camera
O - represents container
. - represents empty cell


k = 1
Min number of cameras: 3.0

Map:
 .  .  .  .  .  . 
 .  O  .  .  .  .
 .  X  O  O  X  O
 .  .  .  .  O  .
 .  .  .  X  O  .
 .  .  .  .  .  .


k = 2
Min number of cameras: 2.0

Map:
 .  .  .  .  .  .
 X  O  .  .  .  .
 .  .  O  O  X  O
 .  .  .  .  O  .
 .  .  .  .  O  .
 .  .  .  .  .  .
\end{verbatim}

\subsection*{6. Wnioski}

Zwiększenie zasięgu kamery ($k$) pozwala na zmniejszenie liczby potrzebnych urządzeń. Dla $k=2$ wystarczyło 2 kamery zamiast 3 (dla $k=1$), co pokazuje korzyści z większego pola widzenia. Wszystkie kontenery zostały skutecznie objęte monitoringiem.

\end{document}