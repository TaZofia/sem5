\documentclass{article}
\usepackage{graphicx}
\usepackage[T1]{fontenc}
\usepackage[utf8]{inputenc}
\usepackage[polish]{babel}
\usepackage{amsmath}

\title{Laboratorium 2 - sprawozdanie}
\author{Zofia Tarchalska}
\date{}

\begin{document}
\maketitle

\section{Zadanie 1}
Optymalizacja dostaw paliwa

\subsection*{1. Opis modelu}

\textbf{Zmienna decyzyjna:}

\begin{itemize}
  \item $x_{s,a}$ - ilość jednostek paliwa dostarczona przez dostawcę $s$ na lotnisko $a$ [jednostki paliwa]
\end{itemize}

\noindent \textbf{Ograniczenia:}

\begin{itemize}
  \item Dostawca nie może dostarczyć więcej niż jego maksymalna dostępność:
  

\[
\sum_{a \in A_s} x_{s,a} \leq \text{supply}_s \quad \forall s \in S
\]

gdzie $S$ to zbiór wszystkich dostawców


\item Każde lotnisko musi otrzymać dokładnie tyle paliwa, ile potrzebuje:
  

\[
\sum_{s \in S_a} x_{s,a} = \text{demand}_a \quad \forall a \in A
\]

gdzie $A$ to zbiór wszystkich lotnisk

\item Dostawy są możliwe tylko dla dozwolonych par $(s,a)$.
\end{itemize}

\noindent \textbf{Funkcja celu:}

\[
\min \sum_{(s,a) \in D} c_{s,a} \cdot x_{s,a}
\]

gdzie:
\begin{itemize}
  \item $c_{s,a}$ - koszt jednostkowy zakupu i dostawy paliwa od dostawcy $s$ na lotnisko $a$ [zł/jednostkę]
  \item $D$ - zbiór dozwolonych par dostawca - lotnisko
\end{itemize}

\subsection*{2. Opis danych i wyników}

Dane wejściowe zostały wczytane z pliku \texttt{data\_ex1.json}. Uwzględniono:
\begin{itemize}
  \item 3 dostawców paliwa o dostępności: 275000, 550000, 660000 jednostek
  \item 4 lotniska o zapotrzebowaniu: 110000, 220000, 330000, 440000 jednostek
  \item Koszty dostawy zależne od pary dostawca - lotnisko
\end{itemize}

\noindent Model został rozwiązany za pomocą solvera GLPK. Uzyskano wyniki:
\begin{verbatim}
Min cost: 8.525e6

Company F1, limit = 275000
Supplier F1 → Airport L1: 0.0
Supplier F1 → Airport L2: 165000.0
Supplier F1 → Airport L3: 0.0
Supplier F1 → Airport L4: 110000.0

F1 delivered 275000.0 in total.

Company F2, limit = 550000
Supplier F2 → Airport L1: 110000.0
Supplier F2 → Airport L2: 55000.0
Supplier F2 → Airport L3: 0.0
Supplier F2 → Airport L4: 0.0

F2 delivered 165000.0 in total.

Company F3, limit = 660000
Supplier F3 → Airport L1: 0.0
Supplier F3 → Airport L2: 0.0
Supplier F3 → Airport L3: 330000.0
Supplier F3 → Airport L4: 330000.0

F3 delivered 660000.0 in total.
\end{verbatim}

\subsection*{Wnioski}
Okazuje się że minimalny koszt wynosi $8 525 000$. Optymalny plan dostaw wskazuje, ile jednostek paliwa należy zakupić od każdego dostawcy i dostarczyć na każde lotnisko, aby spełnić zapotrzebowanie przy minimalnym koszcie. Wyniki są zgodne z ograniczeniami dostępności. Zauważmy, że nie wszytskie firmy muszą dostarczyć paliwo, aby otrzymać optymalne rozwiązanie. Jednak aż dwie firmy F1 i F3 wyczerpały możliwości dostaw (wysłały wszystko co mogły).
\newpage
\section{Zadanie 2}
Optymalizacja produkcji w fabryce
\subsection*{1. Opis modelu}

\begin{itemize}
  \item \textbf{Zmienna decyzyjna:}

  $x_p$ – Liczba kilogramów produktu $p \in P$ wyprodukowana w danym tygodniu [kg]

  \item \textbf{Ograniczenia:}

  \begin{itemize}
    \item Ograniczenia czasowe dla każdej maszyny $m \in M$:
    

\[
    \sum_{p \in P} t_{p,m} \cdot x_p \leq T_m \quad \forall m \in M
    \]


    gdzie:
    \begin{itemize}
      \item $t_{p,m}$ – liczba godzin pracy maszyny $m$ potrzebna do wyprodukowania 1 kg produktu $p$
      \item $T_m$ – maksymalna liczba godzin pracy maszyny $m$ w tygodniu
    \end{itemize}

    \item Ograniczenia popytu dla każdego produktu $p \in P$:
    

\[
    x_p \leq D_p \quad \forall p \in P
    \]


    gdzie:
    \begin{itemize}
      \item $D_p$ – maksymalny tygodniowy popyt na produkt $p$
    \end{itemize}
  \end{itemize}

  \item \textbf{Funkcja celu:}

  

\[
  \max \sum_{p \in P} \left( r_p - c_p - \sum_{m \in M} w_m \cdot t_{p,m} \right) \cdot x_p
  \]



  gdzie:
  \begin{itemize}
    \item $r_p$ – cena sprzedaży 1 kg produktu $p$ [zł/kg]
    \item $c_p$ – koszt materiałowy produkcji 1 kg produktu $p$ [zł/kg]
    \item $w_m$ – koszt pracy maszyny $m$ [zł/godz.]
    \item $t_{p,m}$ – liczba godzin pracy maszyny $m$ na 1 kg produktu $p$
  \end{itemize}
\end{itemize}

\noindent Wynik działania solvera:\\
\begin{verbatim}
Optimal production strategy:
Product P1: 125.0 kg
Product P2: 100.0 kg
Product P3: 150.0 kg
Product P4: 500.0 kg

Max profit: 3632.5 zł
\end{verbatim}

\subsection*{Wnioski}
Model określa optymalną ilość produkcji każdego z produktów, uwzględniając ograniczenia czasowe maszyn oraz maksymalny popyt. Funkcja celu maksymalizuje zysk, który uwzględnia przychód ze sprzedaży, koszt materiałowy oraz koszt pracy maszyn. Okazuje sie, że wszystkie produkty oprócz $P1$ należy wyprodukować w liczbie równej maksymalnemu popytowi. Najwyższy możliwy zysk to 3632.5 zł.
\end{document}