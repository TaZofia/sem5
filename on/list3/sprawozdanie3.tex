\documentclass{article}
\usepackage{subcaption}
\usepackage{graphicx}
\usepackage[T1]{fontenc}
\usepackage[utf8]{inputenc}
\usepackage[polish]{babel}
\usepackage{amsmath}

\setlength{\parindent}{0pt}

\title{Sprawozdanie laboratorium lista 3 - obliczenia naukowe}
\author{\normalsize Zofia Tarchalska, indeks: 279699}
\date{}

\begin{document}
\maketitle

\section*{Zadania 1-3}
Celem zadań jest oprogramowanie funkcji, które będą obliczały miejsca zerowe na 3 różne sposoby. Należało umieścić je w jednym module. Funkcje zwracają:
\begin{itemize}
    \item r - przybliżenie pierwiastka funkcji,
    \item v - wartość f(r),
    \item it - liczba wykonanych operacji,
    \item err - sygnalizacja błędu, różne wartości dla różnych błędów w zależności od zadania i specyficznych ograniczeń każdej z funkcji.
\end{itemize}

\subsection*{Zadanie 1}
Zaimplementowanie funckji rozwiązującej $f(x)=0$ metodą bisekcji. Wymaga ona aby:
\begin{itemize}
    \item funckja $f$ była ciągła na przedziale $[a, b]$, w którym szukamy miejsca zerowego,
    \item oraz żeby miała różne znaki wartości na końcach przedziału, czyli $f(a) \cdot f(b) < 0$.
\end{itemize}

Argumenty funckji:
\begin{itemize}
    \item $f(x)$ - funkcja
    \item $a$, $b$ - początek i koniec przedziału początkowego,
    \item $\delta$ - dokładność w argumentach,
    \item $\epsilon$ - dokładność w wartościach funkcji.
\end{itemize}

\subsubsection*{Sposób działania}
Konsekwentnie dzielimy przedział na pół. Jeśli trafimy na wartość równą 0 - znaleźliśmy rozwiązanie. W przeciwnym razie wybieramy podprzedział, w którym wartości funkcji na krańcach mają przeciwne znaki.

\subsection*{Zadanie 2}
W tym zadaniu zaimplementowana funkcja wylicza miejsca zerowe metodą Newtona. Funkcja musi spełniać poniższe wymagania:
\begin{itemize}
    \item jest określona,
    \item jest ciągła,
    \item pierwsza pochodna $f'(x)$ jest różna od zera. 
\end{itemize}

Argumenty funkcji:
\begin{itemize}
    \item $f(x)$ i $f'(x)$, czyli funkcja oraz jej pochodna,
    \item $x_0$ - przybliżenie początkowe,
    \item $\delta$, $\epsilon$ - dokładności obliczeń, tak samo jak zdefiniowane powyżej,
    \item maxit - maksymalna dopuszczalna liczba operacji.
\end{itemize}

\subsubsection*{Sposób działania}
Jest to metoda iteracyjna. Zaczynamy od przybliżenia początkowego $x_0$ i w każdej iteracji zastępujemy funkcję jej styczną w punkcie $x_i$, wyznaczając przecięcie tej stycznej z osią OX jako nowe przybliżenie $x_{1+1}$.

\subsection*{Zadanie 3}
W tym zadaniu, do obliczenia miejsc zerowych funkcji, należy posłużyć się metodą siecznych. Funkcja musi spełniać poniższe warunki:
\begin{itemize}
    \item jest ciągła,
    \item $f(a) \cdot f(b) < 0$
    \item pierwsza pochodna $f'(x)$ jest różna od zera. Nie istnieje zatem minimum lub maksimum lokalne. Ten warunek gwarantuje nam, iż sieczna nie będzie równoległa do osi OX, co uniemożliwiłoby wyznaczenie jej punktu przecięcia z tą osią.
\end{itemize}

Jako argumenty funkcji podane są:
\begin{itemize}
    \item $f$ - funkcja,
    \item $x_0$, $x_1$ - przybliżenia krańców przedziału, w którym na pewno znajduje się miejsce zerowe,
    \item $\delta$, $\epsilon$ - dokładności obliczeń, tak samo jak zdefiniowane w zadaniu 1,
    \item maxit - maksymalna dopuszczalna liczba iteracji
\end{itemize} 

\subsubsection*{Sposób działania}
Jest to metoda iteracyjna, która przybliża krzywą funkcji na każdym kroku sieczną przechodzącą przez dwa ostatnie przybliżenia i przyjmuje przecięcie tej siecznej z osią OX jako nowe przybliżenie. Działa podobnie jak w zadaniu drugim tylko tutaj pomijamy potrzebę używania jawnego wzoru na pochodną.

Oto wzór:
\[
    x_i = x_{i-1} - f(x_{i-1}) \cdot \frac{x_{i-1} - x_{i-2}}{f(x_{i-1}) - f(x_{i-2})}
\]

\section*{Zadanie 4}
W zadaniu 4 należy przetestować wszystkie 3 metody obliczania miejsc zerowych dla funkcji $f(x) = sin(x) - (\frac{1}{2} \cdot x)^2$. Dla wywołań:
\begin{verbatim}
delta = 1/2 * 10^-5
epsilon = 1/2 * 10^-5

mbisekcji(f, 1.5, 2.0, delta, epsilon)
mstycznych(f, df, 1.5, delta, epsilon, 50)
msiecznych(f, 1.0, 2.0, delta, epsilon, 50)
\end{verbatim}

otrzymujemy wyniki:
\begin{verbatim}
metoda bisekcji
r = 1.9337501525878906
v = 4.772428664701067e-6
it = 17
err = 0

metoda stycznych
r = 1.933753779789742
v = -2.2423316314856834e-8
it = 4
err = 0

metoda siecznych
r = 1.933753644474301
v = 1.564525129449379e-7
it = 4
err = 0  
\end{verbatim}


\subsection*{Wnioski}

Po wrzuceniu tej funkcji do programu Wolfram Alpha otrzymujemy wynik:
\[x \approx 1.93375\]
Widzimy, że wszystkie 3 sposoby są całkiem precyzyjne. Pokrywają się z wynikiem podanym przez Wolframa do 5 miejsc po przecinku. Najdokładniejsze wyniki zwróciła metoda stycznych. W jej przypadku wartość funkcji jest najbliżej zera. Metoda bisekcji potrzebowała aż czterokrotnie więcej iteracji, niż dwie pozostałe, aby znaleźć precyzyjny wynik. Stąd możemy założyć, że w przypadku gdy znamy pochodną funkcji (jest ona łatwa do wyliczenia oraz nie ma złośliwych ekstremów lokalnych) warto posługiwać się metodą stycznych. Metoda siecznych "obchodzi" te problemy i oferuje również całkiem dobre rozwiązanie w podobnej liczbie iteracji. Jeśli jednak na pewno chcemy znaleźć wynik ponad cenę szybkości działania metoda bisekcji będzie pod tym kątem niezawodna.

\section*{Zadanie 5}
Celem zadania 5 jest wyznaczenie wartości zmiennej $x$, dla której przecinają się wykresy funkcji $y = 3x$ i $y = e^x$. Jak wiemy funkcje przecinają się w tym samym punkcie kiedy przyjmują te same wartości dla tych samych argumentów. Czyli przyrównujemy wzory funkcji, aby znaleźć $x$, dla których przyjmują te same wartości.
\[ 3x = e^x \]      
Ostatecznie otrzymujemy takie równanie. 
\[ 3x - e^x = 0 \]
Teraz możemy, zgodnie z poleceniem zadania, metodą bisekcji, policzyć miejsca zerowe funkcji $ g(x) = 3x - e^x $

Do tego potrzebujemy jednak przedziałów, w których znajdują się potencjalne miejsca zerowe.
\begin{figure}[h]
    \centering
    \begin{minipage}{0.45\textwidth}
        \centering
        \includegraphics[width=\linewidth]{3x.png}
        \caption{$y = 3x$}
    \end{minipage}\hfill
    \begin{minipage}{0.45\textwidth}
        \centering
        \includegraphics[width=\linewidth]{ex.png}
        \caption{$y = e^x$}
    \end{minipage}
\end{figure}

Funkcje wyglądają tak jak powyżej. Ze względu na ich kształt widzimy, że będą miały one dwa punkty wspólne.

Dla $x \leq 0$ mamy $3x \leq 0 < e^x$. Wtedy nasza funckja $g(x)$ jest cały czas ujemna. Zatem od nieskończoności do zera nie ma żadnych punktów wspólnych.

Dla $x = 1$ mamy $ e = 2,7... < 3 $, więc $g(x)$ jest dodatnie, natomiast dla $x = 3$ mamy $ e^3 = 20,08..... > 3*3 = 9$, czyli $g(x)$ jest znów ujemne. 

Tym sposobem znalezione zostały dwa przedziały, w których na krańcach funkcja przyjmuje wartości o przeciwnych znakach. Są to $[0, 1]$ oraz $[1, 3]$.

Po wywołaniu funkcji mbisekcji dla tych przedziałów z $\delta = 10^-4$ i $\epsilon = 10^-4$ otrzymujemy poniższe wyniki:
\begin{verbatim}
Przedział [0, 1]
Punkt wspólny x     0.61907958984375
3x - e^x            2.091677592419572e-5
Liczba iteracji     14
Kod błędu           0
Przedział [1, 3]
Punkt wspólny x     1.51214599609375
3x - e^x            -1.7583570236290313e-5
Liczba iteracji     15
Kod błędu           0
\end{verbatim}

\subsection*{Wnioski}
Wyniki wg Wolfram Alpha:
\[
    x \approx 0.619061
\]
\[
    x \approx 1.51213
\]
Dokładność uzykanych przez nas wyników w porównaniu z Wolframem to 4 miejsca po przecinku. Są to wyniki precyzyjne lecz aby je uzyskać potrzebowaliśmy dokładnej analizy problemu. Należało przyjrzeć się przebiegom funkcji i w przybliżeniu określić ich zachowanie na przedziałach. To pozwoliło nam posłużyć się metodą bisekcji i tym razem już precyzyjnie określić wynik. Metoda bisekcji, pod warunkiem spełnienia warunków początkowych, zawsze zapewni nam znalezienie pierwiastka. Jednak, aby wyliczyć wszystkie pierwiastki potrzebujemy posiłkować się wiedzą z analizy matematycznej/rozszerzonej matematyki w liceum. Im węższy przedział uda nam się znaleźć tym barzdiej zaoszczędzimy liczbę iteracji algorytmu.

\section*{Zadanie 6}


\end{document}