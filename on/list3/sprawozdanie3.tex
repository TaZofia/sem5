\documentclass{article}
\usepackage{subcaption}
\usepackage{graphicx}
\usepackage[T1]{fontenc}
\usepackage[utf8]{inputenc}
\usepackage[polish]{babel}
\usepackage{amsmath}

\setlength{\parindent}{0pt}

\title{Sprawozdanie laboratorium lista 3 - obliczenia naukowe}
\author{\normalsize Zofia Tarchalska, indeks: 279699}
\date{}

\begin{document}
\maketitle

\section*{Zadania 1-3}
Celem zadań jest oprogramowanie funkcji, które będą obliczały miejsca zerowe na 3 różne sposoby. Należało umieścić je w jednym module. Funkcje zwracają:
\begin{itemize}
    \item r - przybliżenie pierwiastka funkcji,
    \item v - wartość f(r),
    \item it - liczba wykonanych operacji,
    \item err - sygnalizacja błędu, różne wartości dla różnych błędów w zależności od zadania i specyficznych ograniczeń każdej z funkcji.
\end{itemize}

\subsection*{Zadanie 1}
Zaimplementowanie funkcji rozwiązującej $f(x)=0$ metodą bisekcji. Wymaga ona aby:
\begin{itemize}
    \item funckja $f$ była ciągła na przedziale $[a, b]$, w którym szukamy miejsca zerowego,
    \item oraz żeby miała różne znaki wartości na końcach przedziału, czyli $f(a) \cdot f(b) < 0$.
\end{itemize}

Argumenty funckji:
\begin{itemize}
    \item $f(x)$ - funkcja
    \item $a$, $b$ - początek i koniec przedziału początkowego,
    \item $\delta$ - dokładność w argumentach,
    \item $\epsilon$ - dokładność w wartościach funkcji.
\end{itemize}

\subsubsection*{Sposób działania}
Konsekwentnie dzielimy przedział na pół. Jeśli trafimy na wartość równą 0 - znaleźliśmy rozwiązanie. W przeciwnym razie wybieramy podprzedział, w którym wartości funkcji na krańcach mają przeciwne znaki.

\subsection*{Zadanie 2}
W tym zadaniu zaimplementowana funkcja wylicza miejsca zerowe metodą Newtona. Funkcja musi spełniać poniższe wymagania:
\begin{itemize}
    \item jest określona,
    \item jest ciągła,
    \item pierwsza pochodna $f'(x)$ jest różna od zera. 
\end{itemize}

Argumenty funkcji:
\begin{itemize}
    \item $f(x)$ i $f'(x)$, czyli funkcja oraz jej pochodna,
    \item $x_0$ - przybliżenie początkowe,
    \item $\delta$, $\epsilon$ - dokładności obliczeń, tak samo jak zdefiniowane powyżej,
    \item maxit - maksymalna dopuszczalna liczba operacji.
\end{itemize}

\subsubsection*{Sposób działania}
Jest to metoda iteracyjna. Zaczynamy od przybliżenia początkowego $x_0$ i w każdej iteracji zastępujemy funkcję jej styczną w punkcie $x_i$, wyznaczając przecięcie tej stycznej z osią OX jako nowe przybliżenie $x_{1+1}$.

\subsection*{Zadanie 3}
W tym zadaniu, do obliczenia miejsc zerowych funkcji, należy posłużyć się metodą siecznych. Funkcja musi spełniać poniższe warunki:
\begin{itemize}
    \item jest ciągła,
    \item $f(a) \cdot f(b) < 0$
    \item pierwsza pochodna $f'(x)$ jest różna od zera. Nie istnieje zatem minimum lub maksimum lokalne. Ten warunek gwarantuje nam, iż sieczna nie będzie równoległa do osi OX, co uniemożliwiłoby wyznaczenie jej punktu przecięcia z tą osią.
\end{itemize}

Jako argumenty funkcji podane są:
\begin{itemize}
    \item $f$ - funkcja,
    \item $x_0$, $x_1$ - przybliżenia krańców przedziału, w którym na pewno znajduje się miejsce zerowe,
    \item $\delta$, $\epsilon$ - dokładności obliczeń, tak samo jak zdefiniowane w zadaniu 1,
    \item maxit - maksymalna dopuszczalna liczba iteracji
\end{itemize} 

\subsubsection*{Sposób działania}
Jest to metoda iteracyjna, która przybliża krzywą funkcji na każdym kroku sieczną przechodzącą przez dwa ostatnie przybliżenia i przyjmuje przecięcie tej siecznej z osią OX jako nowe przybliżenie. Działa podobnie jak w zadaniu drugim tylko tutaj pomijamy potrzebę używania jawnego wzoru na pochodną.

Oto wzór:
\[
    x_i = x_{i-1} - f(x_{i-1}) \cdot \frac{x_{i-1} - x_{i-2}}{f(x_{i-1}) - f(x_{i-2})}
\]

\section*{Zadanie 4}
W zadaniu 4 należy przetestować wszystkie 3 metody obliczania miejsc zerowych dla funkcji $f(x) = sin(x) - (\frac{1}{2} \cdot x)^2$. Dla wywołań:
\begin{verbatim}
delta = 1/2 * 10^-5
epsilon = 1/2 * 10^-5

mbisekcji(f, 1.5, 2.0, delta, epsilon)
mstycznych(f, df, 1.5, delta, epsilon, 50)
msiecznych(f, 1.0, 2.0, delta, epsilon, 50)
\end{verbatim}

otrzymujemy wyniki:
\begin{verbatim}
metoda bisekcji
r = 1.9337501525878906
v = 4.772428664701067e-6
it = 17
err = 0

metoda stycznych
r = 1.933753779789742
v = -2.2423316314856834e-8
it = 4
err = 0

metoda siecznych
r = 1.933753644474301
v = 1.564525129449379e-7
it = 4
err = 0  
\end{verbatim}


\subsection*{Wnioski}

Po wrzuceniu tej funkcji do programu Wolfram Alpha otrzymujemy wynik:
\[x \approx 1.93375\]
Widzimy, że wszystkie 3 sposoby są całkiem precyzyjne. Pokrywają się z wynikiem podanym przez Wolframa do 5 miejsc po przecinku. Najdokładniejsze wyniki zwróciła metoda stycznych. W jej przypadku wartość funkcji jest najbliżej zera. Metoda bisekcji potrzebowała aż czterokrotnie więcej iteracji, niż dwie pozostałe, aby znaleźć precyzyjny wynik. Stąd możemy założyć, że w przypadku gdy znamy pochodną funkcji (jest ona łatwa do wyliczenia oraz nie ma złośliwych ekstremów lokalnych) warto posługiwać się metodą stycznych. Metoda siecznych ''obchodzi'' te problemy i oferuje również całkiem dobre rozwiązanie w podobnej liczbie iteracji. Jeśli jednak na pewno chcemy znaleźć wynik ponad cenę szybkości działania metoda bisekcji będzie pod tym kątem niezawodna.

\section*{Zadanie 5}
Celem zadania 5 jest wyznaczenie wartości zmiennej $x$, dla której przecinają się wykresy funkcji $y = 3x$ i $y = e^x$. Jak wiemy funkcje przecinają się w tym samym punkcie kiedy przyjmują te same wartości dla tych samych argumentów. Czyli przyrównujemy wzory funkcji, aby znaleźć $x$, dla których przyjmują te same wartości.
\[ 3x = e^x \]      
Ostatecznie otrzymujemy takie równanie. 
\[ 3x - e^x = 0 \]
Teraz możemy, zgodnie z poleceniem zadania, metodą bisekcji, policzyć miejsca zerowe funkcji $ g(x) = 3x - e^x $

Do tego potrzebujemy jednak przedziałów, w których znajdują się potencjalne miejsca zerowe.
\begin{figure}[h]
    \centering
    \begin{minipage}{0.45\textwidth}
        \centering
        \includegraphics[width=\linewidth]{3x.png}
        \caption{$y = 3x$}
    \end{minipage}\hfill
    \begin{minipage}{0.45\textwidth}
        \centering
        \includegraphics[width=\linewidth]{ex.png}
        \caption{$y = e^x$}
    \end{minipage}
\end{figure}

Funkcje wyglądają tak jak powyżej. Ze względu na ich kształt widzimy, że będą miały one dwa punkty wspólne.

Dla $x \leq 0$ mamy $3x \leq 0 < e^x$. Wtedy nasza funckja $g(x)$ jest cały czas ujemna. Zatem od nieskończoności do zera nie ma żadnych punktów wspólnych.

Dla $x = 1$ mamy $ e = 2,7... < 3 $, więc $g(x)$ jest dodatnie, natomiast dla $x = 3$ mamy $ e^3 = 20,08..... > 3*3 = 9$, czyli $g(x)$ jest znów ujemne. 

Tym sposobem znalezione zostały dwa przedziały, w których na krańcach funkcja przyjmuje wartości o przeciwnych znakach. Są to $[0, 1]$ oraz $[1, 3]$.

Po wywołaniu funkcji mbisekcji dla tych przedziałów z $\delta = 10^-4$ i $\epsilon = 10^-4$ otrzymujemy poniższe wyniki:
\begin{verbatim}
Przedział [0, 1]
Punkt wspólny x     0.61907958984375
3x - e^x            2.091677592419572e-5
Liczba iteracji     14
Kod błędu           0
Przedział [1, 3]
Punkt wspólny x     1.51214599609375
3x - e^x            -1.7583570236290313e-5
Liczba iteracji     15
Kod błędu           0
\end{verbatim}

\subsection*{Wnioski}
Wyniki wg Wolfram Alpha:
\[
    x \approx 0.619061
\]
\[
    x \approx 1.51213
\]
Dokładność uzykanych przez nas wyników w porównaniu z Wolframem to 4 miejsca po przecinku. Są to wyniki precyzyjne lecz aby je uzyskać potrzebowaliśmy dokładnej analizy problemu. Należało przyjrzeć się przebiegom funkcji i w przybliżeniu określić ich zachowanie na przedziałach. To pozwoliło nam posłużyć się metodą bisekcji i tym razem już precyzyjnie określić wynik. Metoda bisekcji, pod warunkiem spełnienia warunków początkowych, zawsze zapewni nam znalezienie pierwiastka. Jednak, aby wyliczyć wszystkie pierwiastki potrzebujemy posiłkować się wiedzą z analizy matematycznej/rozszerzonej matematyki w liceum. Im węższy przedział uda nam się znaleźć tym bardziej zaoszczędzimy liczbę iteracji algorytmu.

\section*{Zadanie 6}
Celem zadania jest znalezienie miejsc zerowych funkcji:
\[f_1(x) = e^{1-x} - 1\]
\[f_2(x) = x \cdot e^{-x}\] 
za pomocą wszystkich trzech metod: bisekcji, stycznych i siecznych. Przyjmujemy:
$\delta = 10^{-5}$ i $\epsilon = 10^{-5}$.

Teraz przeprowadźmy analizę. Łatwo zauważyć, że $f_1(1) = 0$. Funkja ta jest również funkcją malejącą, ponieważ $f_1(x) = e^{1-x}$ jest malejąca, a $-1$ tylko obniża ją jeszcze o $1$. Wobec tego $x = 1$ to jedyne miejsce zerowe $f_1$. Dla funkcji $f_2$ miejscem zerowym jest $x = 0$, ponieważ we wzorze funkcji występuje mnożenie przez $x$. Jest to jedyne miejsce zerowe, ponieważ $e^{-x}$ nigdy nie osiągnie wartości 0. Wystarczy spojrzeć na wykres w zadaniu 5, żeby to zrozumieć. Tam mamy zobrazowany przebieg funkcji $e^x$, która miejsc zerowych nie posiada. Otrzymamy z niej $e^{-x}$ poprzez symetryczne odbicie względem osi OY. Podsumowując:
\[f_1(1) = 0\]
\[f_2(0) = 0\]

\subsection*{Funckja $f_1$}

Dla metody bisekcji:
\begin{verbatim}
Przedział [0.0, 2.0]
x = 1.0
f1(x) = 0.0
Liczba iteracji     1
Kod błędu           0

Przedział [0.0, 3.0]
x = 1.0000019073486328
f1(x) = -1.9073468138230965e-6
Liczba iteracji     19
Kod błędu           0

Przedział [0.0, 10.5]
x = 1.0000011920928955
f1(x) = -1.1920921849650767e-6
Liczba iteracji     21
Kod błędu           0
\end{verbatim}
Wyniki prezentują się następujaco. W przypadku pierwszego przedziału był on symetryczny względem prawdziwego miejsca zerowego. Kolejne były coraz bardziej asymetryczne.

Dla metody stycznych:
\begin{verbatim}
Przybliżenie x = 0.0
x = 0.9999984358892101
f1(x) = 1.5641120130194253e-6
Liczba iteracji     4
Kod błędu           0

Przybliżenie x = 1.0
x = 1.0
f1(x) = 0.0
Liczba iteracji     0
Kod błędu           0

Przybliżenie x = -2.0
x = 0.9999999999251376
f1(x) = 7.486256059507923e-11
Liczba iteracji     7
Kod błędu           0

Przybliżenie x = 1.0e6
x = 1.0e6
f1(x) = -1.0
Liczba iteracji     1
Kod błędu           2
\end{verbatim}

Dla metody siecznych:
\begin{verbatim}
Przybliżenia x1 = 0.0, x2 = 2.0
x = 1.0000017597132702
f1(x) = -1.7597117218937086e-6
Liczba iteracji     6
Kod błędu           0

Przybliżenia x1 = 0.0, x2 = 3.0
x = 0.999999739048799
f1(x) = 2.6095123506486573e-7
Liczba iteracji     9
Kod błędu           0

Przybliżenia x1 = 0.0, x2 = 10.5
x = NaN
f1(x) = NaN
Liczba iteracji     50
Kod błędu           1
\end{verbatim}

\subsection*{Funkcja $f_2$}

Metoda bijekcji:

\begin{verbatim}
Przedział [-1.0, 1.0]
x = 0.0
f2(x) = 0.0
Liczba iteracji     1
Kod błędu           0

Przedział [-1.0, 2.0]
x = 7.62939453125e-6
f2(x) = 7.62933632381113e-6
Liczba iteracji     17
Kod błędu           0

Przedział [-1.0, 9.5]
x = -3.814697265625e-6
f2(x) = -3.814711817567984e-6
Liczba iteracji     17
Kod błędu           0
\end{verbatim}

Metoda stycznych:
\begin{verbatim}
Przybliżenie x = -1.0
x = -3.0642493416461764e-7
f2(x) = -3.0642502806087233e-7
Liczba iteracji     5
Kod błędu           0

Przybliżenie x = 0.0
x = 0.0
f2(x) = 0.0
Liczba iteracji     0
Kod błędu           0

Przybliżenie x = 3.0
x = 14.787436802837927
f2(x) = 5.594878975694858e-6
Liczba iteracji     10
Kod błędu           0

Przybliżenie x = 1.0e6
x = 1.0e6
f2(x) = 0.0
Liczba iteracji     0
Kod błędu           0
\end{verbatim}

Metoda siecznych:
\begin{verbatim}
Przybliżenia x1 = -1.0, x2 = 1.0
x = 1.744165849924562e-8
f2(x) = 1.7441658195034172e-8
Liczba iteracji     18
Kod błędu           0

Przybliżenia x1 = -1.0, x2 = 2.0
x = 14.310428368676307
f2(x) = 8.72393778926339e-6
Liczba iteracji     15
Kod błędu           0

Przybliżenia x1 = -1.0, x2 = 9.5
x = 14.314813369417385
f2(x) = 8.68842855838761e-6
Liczba iteracji     7
Kod błędu           0
\end{verbatim}

\subsection*{Pytania dodatkowe}
Mieliśmy sprawdzić so się stanie, gdy w metodzie Newtona dla $f_1$ wybierzemy miejsce zerowe $x_0 \in (1, \infty]$ oraz $x_0 > 1$ dla $f_2$. Dodatkowo należało się zastanowić czy możemy wybrać $x_0 = 1$ dla $f_2$.

Odpowiedź:
Nie możemy wybrać 1 na miejsce zerowe funkcji $f_2$. Dzieje się tak z powodu wzoru na pochodą. Gdy podstawimy 1 do wzoru $f_2'(x) = e^{-x} \cdot (1 - x)$ otrzymamy 0. To daje dzielenie we wzorze używanym w metodzie Newtona: 
\[x_{n+1} = x_n - \frac{f_2(x_n)}{f_2'(x_n)} = \frac{(x_n)^2}{x_n - 1}\]
Czyli taka opcja musi zostać wykluczona. Dodatkowo gdyby miejsce zerowe było > 1 mielibyśmy sytuację, w której styczne coraz gorzej odwzorowują przebieg funkcji. Dla bardzo dużych $x_0$ może zajść $|f_2(x_0)| < \epsilon$, co spowoduje przyjęcie go za miejsce zerowe. Możemy zaobserowować dla $x = 10^6$. Może się wtedy również stać, że  $|f_2(x_0)| < \epsilon$ co zakończy działanie metody błędem.

Jeśli chodzi o przyjęcie $x_0 > 1$ w przypadku funkcji $f_1$ przy użyciu metody Newtona, jest to również niewskazane dla dużych $x_0$. Gdy miejsce zerowe jest niewiele blisko 1 wtedy metoda zbiega do 1. Jednak dla dużych $x_0$ > 1 pochodna $f_1'(x) = -e^{1-x}$ jest bardzo bliska 0. Wtedy mamy ten sam problem co w poprzednim przypadku. We wzorze:
\[x_{n+1} = x_n - \frac{f_1(x_n)}{f_1'(x_n)} \]
wystąpi dzielenie przez 0. Więc ten przypadek też musimy wykluczyć.

\subsection*{Wnioski}
Podczas stosowania metod wyznaczania pierwiastków należy zachować ostrożność i wcześniej przeanalizować
zachowanie badanych funkcji. Pozwala to uniknąć błędnych rezultatów, które mogą pojawić się nawet dla
pozornie prostych przypadków. Odpowiedni dobór parametrów algorytmu okazuje się kwestią kluczową – niewielka
zmiana warunków początkowych może jedynie wydłużyć czas obliczeń, ale w skrajnych sytuacjach całkowicie
uniemożliwić uzyskanie rozwiązania.

Uzyskane wyniki potwierdzają, że metoda bisekcji jest najbardziej niezawodna. Teoretycznie wystarczy poprawnie
wybrać krańce przedziału $a$ i $b$, aby zawsze dojść do pierwiastka. Jednak brak uwagi ze strony użytkownika
może prowadzić do wykonania ogromnej liczby iteracji bądź błędu.
Próba szybkiego uzyskania przybliżonych wartości pierwiastków przy jednoczesnym wyborze małej dokładności
i szerokiego przedziału może w przypadku „trudnych” funkcji skutkować znacznym błędem.

Warto zauważyć, że samo dostosowanie parametrów $\delta$ i $\epsilon$ nie gwarantuje poprawnego działania
dla metod innych niż bisekcja. Przykładowo, dla funkcji $f_2$ źle dobrany punkt startowy w metodzie Newtona
powoduje, że kolejne przybliżenia oddalają się od rozwiązania. W efekcie można otrzymać bardzo duże wartości
argumentu $x$, dla których funkcja przyjmuje wartości mniejsze od $\varepsilon$, a nawet wpada w zero maszynowe.

\end{document}