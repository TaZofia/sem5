\documentclass{article}
\usepackage{subcaption}
\usepackage{graphicx}
\usepackage[T1]{fontenc}
\usepackage[utf8]{inputenc}
\usepackage[polish]{babel}
\usepackage{amsmath}

\setlength{\parindent}{0pt}

\title{Sprawozdanie laboratorium lista 3 - obliczenia naukowe}
\author{\normalsize Zofia Tarchalska, indeks: 279699}
\date{}

\begin{document}
\maketitle

\section{Zadania 1-3}
Celem zadań jest oprogramowanie funkcji, które będą obliczały miejsca zerowe na 3 różne sposoby. Należało umieścić je w jednym module. Funkcje zwracają:
\begin{itemize}
    \item r - przybliżenie pierwiastka funkcji,
    \item v - wartość f(r),
    \item it - liczba wykonanych operacji,
    \item err - sygnalizacja błędu, różne wartości dla różnych błędów w zależności od zadania i specyficznych ograniczeń każdej z funkcji.
\end{itemize}

\subsection{Zadanie 1}
Zaimplementowanie funckji rozwiązującej $f(x)=0$ metodą bisekcji. Wymaga ona aby:
\begin{itemize}
    \item funckja $f$ była ciągła na przedziale $[a, b]$, w którym szukamy miejsca zerowego,
    \item oraz żeby miała różne znaki wartości na końcach przedziału, czyli $f(a) \cdot f(b) < 0$.
\end{itemize}

Argumenty funckji:
\begin{itemize}
    \item $f(x)$ - funkcja
    \item $a$, $b$ - początek i koniec przedziału początkowego,
    \item $\delta$ - dokładność w argumentach,
    \item $\epsilon$ - dokładność w wartościach funkcji.
\end{itemize}

\subsubsection*{Sposób działania}
Konsekwentnie dzielimy przedział na pół. Jeśli trafimy na wartość równą 0 - znaleźliśmy rozwiązanie. W przeciwnym razie wybieramy podprzedział, w którym wartości funkcji na krańcach mają przeciwne znaki.

\subsection{Zadanie 2}
W tym zadaniu zaimplementowana funkcja wylicza miejsca zerowe metodą Newtona. Funkcja musi spełniać poniższe wymagania:
\begin{itemize}
    \item jest określona,
    \item jest ciągła,
    \item pierwsza pochodna $f'(x)$ jest różna od zera. 
\end{itemize}

Argumenty funkcji:
\begin{itemize}
    \item $f(x)$ i $f'(x)$, czyli funkcja oraz jej pochodna,
    \item $x_0$ - przybliżenie początkowe,
    \item $\delta$, $\epsilon$ - dokładności obliczeń, tak samo jak zdefiniowane powyżej,
    \item maxit - maksymalna dopuszczalna liczba operacji.
\end{itemize}

\subsubsection*{Sposób działania}
Jest to metoda iteracyjna. Zaczynamy od przybliżenia początkowego $x_0$ i w każdej iteracji zastępujemy funkcję jej styczną w punkcie $x_i$, wyznaczając przecięcie tej stycznej z osią OX jako nowe przybliżenie $x_{1+1}$.

\subsection{Zadanie 3}
W tym zadaniu, do obliczenia miejsc zerowych funkcji, należy posłużyć się metodą siecznych. Funkcja musi spełniać poniższe warunki:
\begin{itemize}
    \item jest ciągła,
    \item $f(a) \cdot f(b) < 0$
    \item pierwsza pochodna $f'(x)$ jest różna od zera. Nie istnieje zatem minimum lub maksimum lokalne. Ten warunek gwarantuje nam, iż sieczna nie będzie równoległa do osi OX, co uniemożliwiłoby wyznaczenie jej punktu przecięcia z tą osią.
\end{itemize}

Jako argumenty funkcji podane są:
\begin{itemize}
    \item $f$ - funkcja,
    \item $x_0$, $x_1$ - przybliżenia krańców przedziału, w którym na pewno znajduje się miejsce zerowe,
    \item $\delta$, $\epsilon$ - dokładności obliczeń, tak samo jak zdefiniowane w zadaniu 1,
    \item maxit - maksymalna dopuszczalna liczba iteracji
\end{itemize} 

\subsubsection*{Sposób działania}
Jest to metoda iteracyjna, która przybliża krzywą funkcji na każdym kroku sieczną przechodzącą przez dwa ostatnie przybliżenia i przyjmuje przecięcie tej siecznej z osią OX jako nowe przybliżenie. Działa podobnie jak w zadaniu drugim tylko tutaj pomijamy potrzebę używania jawnego wzoru na pochodną.

Oto wzór:
\[
    x_i = x_{i-1} - f(x_{i-1}) \cdot \frac{x_{i-1} - x_{i-2}}{f(x_{i-1}) - f(x_{i-2})}
\]









\end{document}