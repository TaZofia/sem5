\documentclass{article}
\usepackage{subcaption}
\usepackage{graphicx}
\usepackage[T1]{fontenc}
\usepackage[utf8]{inputenc}
\usepackage[polish]{babel}
\usepackage{amsmath}
\usepackage{amsfonts}
\usepackage{float}

\setlength{\parindent}{0pt}

\title{Sprawozdanie laboratorium lista 4 \\Obliczenia naukowe}
\author{\normalsize Zofia Tarchalska, indeks: 279699}
\date{}

\begin{document}
\maketitle
\section*{Wstęp do zadań 1-4}
W pierwszych trzech zadaniach należy zaimplementować algortymy dotyczące interpolacji wielomianowej.

Interpolacja wielomianowa jest jednym z podstawowych narzędzi analizy numerycznej, pozwalającym na przybliżanie funkcji za pomocą wielomianów przechodzących przez zadane punkty. Szczególnie użyteczna jest postać Newtona, w której współczynniki wyznaczane są za pomocą ilorazów różnicowych. Ta reprezentacja jest wygodna obliczeniowo i umożliwia łatwe dodawanie nowych węzłów bez konieczności ponownego wyznaczania całego wielomianu.

W sprawozdaniu rozważymy trzy powiązane ze sobą algorytmy:
\begin{enumerate}
    \item Obliczanie ilorazów różnicowych na podstawie wartości funkcji w węzłach interpolacji.
    \item Wyznaczanie wartości wielomianu Newtona w punkcie $t$ za pomocą uogólnionego algorytmu Hornera, działającego w czasie liniowym $O(n)$.
    \item Przekształcanie wielomianu Newtona do postaci naturalnej $a_0 + a_1 x + \dots + a_n x^n$, co wymaga czasu $O(n^2)$.
\end{enumerate}


\section*{Zadanie 1}

\subsection*{Algorytm obliczania ilorazów różnicowych}

\subsubsection*{Wejście}
\begin{itemize}
    \item Wektor węzłów $x = [x_0, x_1, \dots, x_n]$ długości $n+1$.
    \item Wektor wartości funkcji $f = [f(x_0), f(x_1), \dots, f(x_n)]$ długości $n+1$.
\end{itemize}

\subsubsection*{Wyjście}
\begin{itemize}
    \item Wektor $fx = [f[x_0], f[x_0,x_1], \dots, f[x_0,\dots,x_n]]$ długości $n+1$, zawierający kolejne ilorazy różnicowe.
\end{itemize}

\subsubsection*{Idea działania}
Algorytm oblicza współczynniki dla wielomianu interpolacyjnego Newtona. 
Zamiast tworzyć tablicę dwuwymiarową, wszystkie obliczenia wykonuje w jednym wektorze, nadpisując jego wartości w kolejnych krokach. 
Pierwszy element wektora wynikowego jest wartością funkcji w pierwszym węźle, a kolejne elementy powstają poprzez iteracyjne obliczanie ilorazów różnicowych wyższych rzędów.

\subsubsection*{Opis kroków}
\begin{enumerate}
    \item Utwórz wektor $fx$ długości $n+1$ i skopiuj do niego wartości funkcji $f(x_i)$.
    \item Ustal $fx[1] = f(x_0)$.
    \item Dla każdego rzędu $j = 2, 3, \dots, n+1$:
    \begin{itemize}
        \item Przejdź po wektorze od końca do początku.
        \item Zastąp element $fx[i]$ wartością:
        

\[
        fx[i] \gets \frac{fx[i] - fx[i-1]}{x[i] - x[i-j+1]}
        \]


        \item Pierwszy element z danego rzędu odpowiada współczynnikowi Newtona i pozostaje w $fx[j]$.
    \end{itemize}
    \item Po zakończeniu pętli wektor $fx$ zawiera wszystkie ilorazy różnicowe.
\end{enumerate}

\subsubsection*{Złożoność}
\begin{itemize}
    \item Czasowa: $O(n^2)$, ponieważ algorytm wykonuje podwójną pętlę.
    \item Pamięciowa: $O(n)$, ponieważ używany jest tylko jeden wektor długości $n+1$.
\end{itemize}

\section*{Zadanie 2}
\subsection*{Algorytm obliczania wartości wielomianu Newtona}

\subsubsection*{Wejście}
\begin{itemize}
    \item Wektor węzłów $x = [x_0, x_1, \dots, x_n]$ długości $n+1$.
    \item Wektor ilorazów różnicowych $fx = [f[x_0], f[x_0,x_1], \dots, f[x_0,\dots,x_n]]$ długości $n+1$.
    \item Punkt $t \in \mathbb{R}$, w którym należy obliczyć wartość wielomianu interpolacyjnego Newtona.
\end{itemize}

\subsubsection*{Wyjście}
\begin{itemize}
    \item Wartość wielomianu Newtona $N_n(t)$ w punkcie $t$.
\end{itemize}

\subsubsection*{Idea działania}
Ponieważ wielomian Newtona jest postaci:

\begin{multline}
N_n(t) = f[x_0] + f[x_0,x_1](t-x_0) + f[x_0,x_1,x_2](t-x_0)(t-x_1) \\
+ \dots + f[x_0,\dots,x_n](t-x_0)(t-x_1)\dots(t-x_{n-1})
\end{multline}

Bezpośrednie obliczanie wymagałoby mnożenia coraz dłuższych iloczynów i dodawania kolejnych wyrazów, co prowadzi do kosztu rzędu $O(n^2)$. Uogólniony algorytm Hornera pozwala uniknąć tego problemu, przekształcając obliczenia w prostą procedurę rekurencyjnego „zwijania” wielomianu. 

Zaczynamy od najwyższego współczynnika (ilorazu różnicowego rzędu $n$) i cofamy się w dół, w każdej iteracji doklejając czynnik $(t - x_i)$ oraz dodając współczynnik niższego rzędu. Dzięki temu wielomian jest budowany krok po kroku, bez potrzeby jawnego rozwijania wszystkich iloczynów. 

\subsubsection*{Opis kroków}
\begin{enumerate}
    \item Ustaw zmienną pomocniczą $w \gets fx[n+1]$, czyli najwyższy współczynnik.
    \item Dla $i = n, n-1, \dots, 1$ wykonaj:
    

\[
    w \gets w \cdot (t - x[i]) + fx[i].
    \]


    \item Po zakończeniu pętli zmienna $w$ zawiera wartość wielomianu Newtona w punkcie $t$.
\end{enumerate}

\subsubsection*{Złożoność}
\begin{itemize}
    \item Czasowa: $O(n)$, ponieważ wykonujemy dokładnie $n$ kroków.
    \item Pamięciowa: $O(1)$, ponieważ używana jest tylko jedna zmienna pomocnicza $w$.
\end{itemize}

\section*{Zadanie 3}
\subsection*{Algorytm przekształcania wielomianu Newtona do postaci naturalnej}

\subsubsection*{Wejście}
\begin{itemize}
    \item Wektor węzłów $x = [x_0, x_1, \dots, x_n]$ długości $n+1$.
    \item Wektor ilorazów różnicowych $fx = [f[x_0], f[x_0,x_1], \dots, f[x_0,\dots,x_n]]$ długości $n+1$.
\end{itemize}

\subsubsection*{Wyjście}
\begin{itemize}
    \item Wektor współczynników postaci naturalnej $a = [a_0, a_1, \dots, a_n]$, gdzie
    

\[
    N_n(x) = a_0 + a_1 x + a_2 x^2 + \dots + a_n x^n.
    \]


\end{itemize}

\subsubsection*{Idea działania}
Wielomian Newtona zapisany jest w postaci iloczynowej z ilorazami różnicowymi jako współczynnikami. Aby uzyskać jego postać naturalną, należy kolejno rozwijać czynniki $(x - x_i)$ i aktualizować współczynniki wielomianu. Algorytm działa iteracyjnie: zaczyna od najwyższego współczynnika $fx[n+1]$, a następnie w każdej iteracji „dokleja” kolejny czynnik $(x - x_k)$, przesuwając współczynniki w dół i dodając odpowiedni iloraz różnicowy. Dzięki temu po $n$ krokach powstaje pełny wektor współczynników postaci naturalnej.  

\subsubsection*{Opis kroków}
\begin{enumerate}
    \item Utwórz wektor $a$ długości $n+1$ i wypełnij zerami.
    \item Ustaw $a[n+1] \gets fx[n+1]$ (najwyższy współczynnik Newtona).
    \item Dla $k = n, n-1, \dots, 1$:
    \begin{enumerate}
        \item Dla $j = k, k-1, \dots, 1$ wykonaj:
        

\[
        a[j] \gets a[j] - x[k] \cdot a[j+1].
        \]


        \item Następnie zaktualizuj:
        

\[
        a[k] \gets a[k] + fx[k].
        \]


    \end{enumerate}
    \item Po zakończeniu pętli wektor $a$ zawiera współczynniki postaci naturalnej.
\end{enumerate}

\subsubsection*{Złożoność}
\begin{itemize}
    \item Czasowa: $O(n^2)$, ponieważ każda z $n$ iteracji wymaga aktualizacji do $n$ współczynników.
    \item Pamięciowa: $O(n)$, ponieważ przechowywany jest tylko wektor wynikowy $a$.
\end{itemize}

\section*{Zadanie 4}
\subsection*{Algorytm rysowania wielomianu interpolacyjnego Newtona}

\subsubsection*{Wejście}
\begin{itemize}
    \item Funkcja $f(x)$ zadana jako anonimowa funkcja.
    \item Przedział interpolacji $[a,b]$.
    \item Stopień wielomianu interpolacyjnego $n$.
\end{itemize}

\subsubsection*{Wyjście}
\begin{itemize}
    \item Wykres przedstawiający na jednym rysunku:
    \begin{enumerate}
        \item interpolowaną funkcję $f(x)$,
        \item wielomian interpolacyjny Newtona $N_n(x)$.
    \end{enumerate}
\end{itemize}

\subsubsection*{Idea działania}
Algorytm konstruuje wielomian interpolacyjny Newtona na podstawie wartości funkcji w wybranych węzłach interpolacji. Węzły są równoodległe w przedziale $[a,b]$. Najpierw obliczane są wartości funkcji w tych punktach, a następnie ilorazy różnicowe, które stanowią współczynniki postaci Newtona. Do obliczania wartości wielomianu w dowolnym punkcie wykorzystuje się funkcję \texttt{warNewton}, która działa w czasie liniowym $O(n)$.  

Aby narysować przebieg funkcji i wielomianu, algorytm tworzy gęstą siatkę punktów w przedziale $[a,b]$, oblicza w nich wartości zarówno funkcji $f(x)$, jak i interpolantu $N_n(x)$, a następnie rysuje oba przebiegi na jednym wykresie. Dzięki temu można wizualnie porównać jakość interpolacji.

\subsubsection*{Opis kroków}
\begin{enumerate}
    \item Utwórz wektory $x$ i $y$ długości $n+1$.
    \item Wyznacz równoodległe węzły $x_k = a + k \cdot h$, gdzie $h = \frac{b-a}{n}$, oraz oblicz wartości $y_k = f(x_k)$.
    \item Oblicz współczynniki Newtona $c$ za pomocą funkcji \texttt{ilorazyRoznicowe}.
    \item Utwórz gęstą siatkę punktów $xs$ w przedziale $[a,b]$.
    \item Dla każdego punktu $xs[i]$ oblicz:
    \begin{itemize}
        \item wartość wielomianu Newtona $poly[i] = warNewton(x,c,xs[i])$,
        \item wartość funkcji $func[i] = f(xs[i])$.
    \end{itemize}
    \item Narysuj wykres funkcji i wielomianu interpolacyjnego na tym samym rysunku.
\end{enumerate}

\subsubsection*{Złożoność - zastanowić się jeszcze}
\begin{itemize}
    \item Obliczanie ilorazów różnicowych: $O(n^2)$.
    \item Obliczanie wartości wielomianu w siatce: $O(n \cdot m)$, gdzie $m$ to liczba punktów siatki.
    \item Pamięciowa: $O(n+m)$, na przechowywanie węzłów, wartości funkcji i wyników interpolacji.
\end{itemize}

\section*{Zadanie 5}
Celem zadania było wykorzystanie napisanej wcześniej funkcji \texttt{rysujNnfx(f, a, b, n, wezly)} do zaprezentowania na wykresie wyników interpolacji dla funkcji:
\begin{itemize}
    \item $f(x) = e^x$ na przedziale $[0, 1]$
    \item $f(x) = x^2 \cdot sin(x)$ na przedziale $[-1, 1]$
\end{itemize}
Stopnie wielomianu: 5, 10, 15. Węzły - równoodległe.

Poniżej znajdują się wykresy dla $f(x) = e^x$.
\begin{figure}[H]
    \centering
    \includegraphics[width=0.6\textwidth]{ex5_results/ex5_a_rownoodlegle_n5.png}
    \caption{n = 5}
\end{figure}
\begin{figure}[H]
    \centering
    \includegraphics[width=0.6\textwidth]{ex5_results/ex5_a_rownoodlegle_n10.png}
    \caption{n = 10}
\end{figure}
\begin{figure}[H]
    \centering
    \includegraphics[width=0.6\textwidth]{ex5_results/ex5_a_rownoodlegle_n15.png}
    \caption{n = 15}
\end{figure}

A oto wyniki dla $f(x) = x^2 \cdot sin(x)$
\begin{figure}[H]
    \centering
    \includegraphics[width=0.6\textwidth]{ex5_results/ex5_b_rownoodlegle_n5.png}
    \caption{n = 5}
\end{figure}
\begin{figure}[H]
    \centering
    \includegraphics[width=0.6\textwidth]{ex5_results/ex5_b_rownoodlegle_n10.png}
    \caption{n = 10}
\end{figure}
\begin{figure}[H]
    \centering
    \includegraphics[width=0.6\textwidth]{ex5_results/ex5_b_rownoodlegle_n15.png}
    \caption{n = 15}
\end{figure}

\subsection*{Wnioski}
Możemy zauważyć, że obydwie testowane funckje udaje się bardzo dokładnie interpolować. Wartości wielomianów interpolacyjnych dla każdego ze sprawdzanych stopni (5, 10, 15) niemal praktycznie idealnie pokrywają się z wartościami funkcji na całym przedziale.

\section*{Zadanie 6}
W tym zadaniu również należało użyć \texttt{rysujNnfx(f, a, b, n, wezly)} jednak tym razem dla funkcji:
\begin{itemize}
    \item $f(x) = |x|,\; x \in [-1,1]$
    \item $f(x) = \dfrac{1}{1+x^2},\; x \in [-5,5]$
\end{itemize}
Stopnie wielomianu, tak jak w poprzednim zadaniu: 5, 10, 15. Węzły - równoodległe i Czebyszewa.
\newpage
Wyniki dla $f(x) = |x|$
\begin{figure}[H]
    \centering
    \begin{minipage}{0.45\textwidth}
        \centering
        \includegraphics[width=\linewidth]{ex6_results/ex6_a_czebyszew_n5.png}
        \caption{czebyszew, n = 5}
    \end{minipage}\hfill
    \begin{minipage}{0.45\textwidth}
        \centering
        \includegraphics[width=\linewidth]{ex6_results/ex6_a_rownoodlegle_n5.png}
        \caption{równoodległe, n = 5}
    \end{minipage}

    \vspace{0.5cm}
    
    \begin{minipage}{0.45\textwidth}
        \centering
        \includegraphics[width=\linewidth]{ex6_results/ex6_a_czebyszew_n10.png}
        \caption{czebyszew, n = 10}
    \end{minipage}\hfill
    \begin{minipage}{0.45\textwidth}
        \centering
        \includegraphics[width=\linewidth]{ex6_results/ex6_a_rownoodlegle_n10.png}
        \caption{równoodległe, n = 10}
    \end{minipage}

    \vspace{0.5cm}

    \begin{minipage}{0.45\textwidth}
        \centering
        \includegraphics[width=\linewidth]{ex6_results/ex6_a_czebyszew_n15.png}
        \caption{czebyszew, n = 15}
    \end{minipage}\hfill
    \begin{minipage}{0.45\textwidth}
        \centering
        \includegraphics[width=\linewidth]{ex6_results/ex6_a_rownoodlegle_n15.png}
        \caption{równoodległe, n = 15}
    \end{minipage}
\end{figure}

\newpage
Wyniki dla $f(x) = \dfrac{1}{1+x^2}$
\begin{figure}[H]
    \centering
    \begin{minipage}{0.45\textwidth}
        \centering
        \includegraphics[width=\linewidth]{ex6_results/ex6_b_czebyszew_n5.png}
        \caption{czebyszew, n = 5}
    \end{minipage}\hfill
    \begin{minipage}{0.45\textwidth}
        \centering
        \includegraphics[width=\linewidth]{ex6_results/ex6_b_rownoodlegle_n5.png}
        \caption{równoodległe, n = 5}
    \end{minipage}

    \vspace{0.5cm}
    
    \begin{minipage}{0.45\textwidth}
        \centering
        \includegraphics[width=\linewidth]{ex6_results/ex6_b_czebyszew_n10.png}
        \caption{czebyszew, n = 10}
    \end{minipage}\hfill
    \begin{minipage}{0.45\textwidth}
        \centering
        \includegraphics[width=\linewidth]{ex6_results/ex6_b_rownoodlegle_n10.png}
        \caption{równoodległe, n = 10}
    \end{minipage}

    \vspace{0.5cm}

    \begin{minipage}{0.45\textwidth}
        \centering
        \includegraphics[width=\linewidth]{ex6_results/ex6_b_czebyszew_n15.png}
        \caption{czebyszew, n = 15}
    \end{minipage}\hfill
    \begin{minipage}{0.45\textwidth}
        \centering
        \includegraphics[width=\linewidth]{ex6_results/ex6_b_rownoodlegle_n15.png}
        \caption{równoodległe, n = 15}
    \end{minipage}
\end{figure}
W przeciwieństwie do poprzedniego zadania funckje nie są zbyt dokładnie interpolowane. W przypadku $|x|$ problelem jest nieróżniczkowalność takiej funkcji. Widzimy, że największy problem jest z dokładnym odwzorowaniem ostrego czubka na wykresie wartości bezwzględnej z x. Dla drugiej funkcji zauważamy zwiększenie rozbieżności między funkcją a interpolacją. Jest to tak zwane zjawisko Rungego. Pojawia się ono tylko w przypadku węzłów równoodległych. Zwiększyć dokładność mogłoby pomóc zwiększenie liczby węzłów na krańcach przedziału.

\subsection*{Wnioski}
Okazuje się, że interpolacja lepiej sprawdza się w przypadku gładkich funkcji np. takich jak w zadaniu 5. Jednak nawet gdy kształt funkcji wydaje się być pożądany, tak jak w podpunkcie b w zadaniu 6, może się okazać, że trzeba również rozsądnie przemyśleć rozstawienie węzłów interpolacji. Zwiększanie stopnia wielomianu w tym wypadku nic nie pomoże, a dokładniejsze wyniki uzyskamy korzystając z innej metody wyznaczania węzłów (w przypadku zadania 6 jest to metoda Czebyszewa).
\end{document}