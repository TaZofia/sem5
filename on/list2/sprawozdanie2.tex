\documentclass{article}
\usepackage{graphicx}
\usepackage[T1]{fontenc}
\usepackage[utf8]{inputenc}
\usepackage[polish]{babel}
\usepackage{amsmath}
\usepackage[a4paper, left=1.5cm, right=1.5cm]{geometry}
\usepackage{breqn}

\title{Sprawozdanie laboratorium lista 2 - obliczenia naukowe}
\author{\normalsize Zofia Tarchalska}
\date{}

\begin{document}
\maketitle


\section{Zadanie 1}
Celem zadania było powtórzenie zadania 5 z listy 1, ale z lekko zmodyfikowanymi danymi. Dokładnie chodziło o to aby usunąć ostatnią 9 z czwartej współrzędnej wektora x i ostatnią 7 z piątej współrzędnej. Wektory te prezentują się następująco:\\
\noindent $x = [2.718281828, -3.141592654, 1.414213562, 0.5772156649, 0.3010299957]$\\
\noindent $y = [1486.2497, 878366.9879, -22.37492, 4773714.647, 0.000185049]$\\

\noindent Po zmianie:\\
$x = [2.718281828, -3.141592654, 1.414213562, 0.577215664, 0.301029995]$\\

\noindent Po lewej: poprzednie uzyskane wyniki. Po prawej: nowe wyniki\\

\noindent
\begin{minipage}{0.48\textwidth}
\begin{verbatim}
Float32
real: -1.00657107000000e-11
forward: -0.4999443
backward: -0.4543457
biggest_to_smallest: -0.5
smallest_to_biggest: -0.5

Float64
real: -1.00657107000000e-11
forward: 1.0251881368296672e-10
backward: -1.5643308870494366e-10
biggest_to_smallest: 0.0
smallest_to_biggest: 0.0
\end{verbatim}
\end{minipage}
\hfill
\begin{minipage}{0.48\textwidth}
\begin{verbatim}
Float32
-----------------------------
forward: -0.4999443
backward: -0.4543457
biggest_to_smallest: -0.5
smallest_to_biggest: -0.5

Float64
-----------------------------
forward: -0.004296342739891585
backward: -0.004296342998713953
biggest_to_smallest: -0.004296342842280865
smallest_to_biggest: -0.004296342842280865
\end{verbatim}
\end{minipage}

\noindent \\\\Można zauważyć, że jeśli chodzi o arytmetykę single wyniki w ogóle nie różnią się od tych uzyskanych w poprzedniej próbie. Dzieje się tak, ponieważ usuwane cyfry są na gracnicy precyzji. Inaczej jest w przypadku double, w tej arytmetyce otrzymujemy różne rezultaty. 

\subsection*{Wnioski} Okazuje się, że w przypadku arytmetyki Float64 niewielka zmiana na 10 miejscu po przecinku powoduje zmianę wyniku o 7 rzędów wielkości. Najpierw wynik był rzędu $10^{-10}$ żeby następnie wzrosnąć rzędu $10_{-3}$ (dla sposobu forawrd i backward, ponieważ są to te bardziej precyzyjne sposoby). Widzimy zatem, że zadanie jest źle uwraunkowane (mała zmiana w danych powoduje duże zmiany wyniku).

\section{Zadanie 2}
W zadaniu należy narysować wykres funkcji $f(x) = e^xln(1+e^{-x})$ w co najmniej dwóch różnych programach do wizualizacji. Następnie trzeba policzyć jej granicę i porównać z otrzymanymi wykresami.\\
\noindent
\begin{minipage}{0.48\textwidth}
    \begin{center}
        \includegraphics[width=1\textwidth]{pyplot_fx.png}
    \end{center}
\end{minipage}
\hfill
\begin{minipage}{0.48\textwidth}
    \begin{center}
        \includegraphics[width=1\textwidth]{wolfram_fx.png}
    \end{center}
\end{minipage}

\noindent Po lewej widnieje wykres wygenerowany za pomocą PyPlot, a po prawej w WolframAlpha.

\noindent Teraz ręcznie policzmy granicę funkcji w nieskończoności:
\[
 \lim_{x \to \infty } e^x \cdot ln(1 + e^{-x}) = \lim_{x \to \infty} \frac {-e^{-x}}{(1 + e^{-x}) \cdot (-e ^{-x})} = \lim_{x \to \infty} \frac{1}{1 + e^{-x}} = 1
\]

\noindent Jak widać rzeczywisty przebieg funckji różni się od tego co zwracają nam obydwa programy.
\subsection*{Wnioski} 
Znów zadanie charakteryzuje się silnym uwarunkowaniem numerycznym. Niedokładności wynikające z ograniczonej percyzji, prowadzą do odchyleń wartości funkcji od prawidłowego wyniku, szczególnie w zakresie $x \in [30, 36]$. Dla argumentów $>36$ funkcja zbiega do 0, ponieważ zachodzi przybliżenie $ 1 + e^{-x} \approx 1$ zatem $ln(1 + e^{-x}) \approx 0$. Jest to sprzeczne z rzeczywistym przebiegiem funkcji i wyznaczoną granicą analityczną. Zaburzenie występuje w obydwu programach zewnętrznych, co oznacza złe uwarunkowanie zadania.

\section{Zadanie 3}
Zadanie polegało na rozwiązaniu układu równań liniowych dwoma różnymi sposobami oraz porównaniu ich pod kątem policzonych błędów względnych.


\noindent \\ Mamy równanie liniowe postaci: $Ax = b$, gdzie: \\
\begin{itemize}
\item A to macierz współczynników. Generujemy ją na dwa sposoby:\\
    \begin{itemize}
        \item $A = H_n$, gdzie $H_n$ jest macierzą Hilberta stopnia n 
        \item $A = R_n$, gdzie $R_n$ to losowa macierz stopnia n z podanym wskaźnikiem uwarunkowania c
    \end{itemize}
\item b to wektor prawych stron
\end{itemize}

\noindent Układ ten będziemy rozwiązywać dwoma metodami:
\begin{itemize}
    \item metodą eliminacji Gauss'a - $x = A \backslash b$
    \item metodą z macierza odwrotną - $x = inv(A) * b$
\end{itemize}

\noindent Nasz dokładny $x$ to $x = (1, ..., 1)^T$. Z jego pomocą będziemy obliczać błąd względny. Na nastęnych stronach zamieszczone zostają wartości zwracane przez funkcje \texttt{rank(A)} i \texttt{cond(A)} oraz policzone błędy względne dla obu metod.
\newpage
\begin{verbatim}
Macierz Hilberta

n   cond(A)                  rank(A) error Gauss              error inv
1   1.0                      1       0.0                      0.0
2   19.28147006790397        2       5.661048867003676e-16    1.4043333874306803e-15   
3   524.0567775860644        3       8.022593772267726e-15    0.0
4   15513.73873892924        4       4.137409622430382e-14    0.0
5   476607.2502422687        5       1.6828426299227195e-12   3.3544360584359632e-12   
6   1.49510586424659e7       6       2.618913302311624e-10    2.0163759404347654e-10   
7   4.753673565921816e8      7       1.2606867224171548e-8    4.713280397232037e-9
8   1.5257575538072489e10    8       6.124089555723088e-8     3.07748390309622e-7
9   4.931537556012197e11     9       3.8751634185032475e-6    4.541268303176643e-6
10  1.602441350036382e13     10      8.67039023709691e-5      0.0002501493411824886
11  5.222703245009594e14     10      0.00015827808158590435   0.007618304284315809
12  1.760619121841585e16     11      0.13396208372085344      0.258994120804705
13  3.1905581877988255e18    11      0.11039701117868264      5.331275639426837
14  9.27636978936766e17      11      1.4554087127659643       8.71499275104814
15  3.67568286586649e17      12      4.696668350857427        7.344641453111494
16  7.063115212292111e17     12      54.15518954564602        29.84884207073541
17  8.07124989431416e17      12      13.707236683836307       10.516942378369349
18  1.4135073701749765e18    12      10.257619124632317       24.762070989128866
19  5.190132496359103e18     13      102.15983486270827       109.94550732878284
20  1.3193976166344822e18    13      108.31777346206205       114.34403152557572
21  3.2903033202156175e18    13      44.089455838364245       34.52041154914292
22  8.482350008309597e18     13      17.003425713362045       102.60161611995359
23  6.101209031674573e17     13      25.842511917947366       22.272314298730727
24  1.8162451419244399e19    13      39.638573210187644       43.34914763015038
25  1.3309197553221074e18    13      7.095757204652332        21.04404299195525
26  7.779515179373411e18     14      63.80426636186403        100.78434642499187       
27  4.28683702161786e18      14      27.43309009053957        35.68974530952139
28  5.937872779302876e18     14      276.91498822022265       290.1167291705239
29  8.277434084408434e18     14      60.095450394724104       43.40383683199056
30  3.8719824664564173e18    14      24.80615905441871        59.97231132227779
31  9.796434738176467e18     14      21.45662601984968        23.74575780277118
32  4.2803982785172644e18    14      36.582441571177284       67.4381226943068
33  1.1705168465593727e19    14      37.556822732776205       32.88969741379979
34  5.546235957952042e18     14      88.87380459381126        95.99116506490785
35  2.552419613144824e19     14      31.166902974731222       36.723963451169304
36  4.227992561870757e18     15      15.563379312608328       19.599011323097056
37  5.859007350289631e18     15      13.974714130452178       16.39248770656996
38  8.652991891691735e18     15      72.12122789133323        95.5655782183542
39  1.8383449979886094e19    15      118.2033650158989        263.5309838641091
40  6.581732387647914e18     15      23.926484807638683       140.97274594056717
41  1.5426903357896567e19    15      41.348771577098454       40.75749340255354
42  2.9056333619025285e19    15      229.6423260398746        333.75226335487844
43  1.4838416581923312e19    15      53.18930954995267        54.52704305417691
44  2.6895334840373182e19    15      124.67413636996756       94.88356401052424
45  1.214705872715781e19     15      244.58124814685374       179.92316617880468
46  1.5097027936171698e19    15      69.14584939886464        109.17112219679052
47  1.9943467382012723e20    15      41.43803149349301        83.82203728470039
48  1.0925283248003965e19    15      58.952689545073156       156.78973560359313
49  6.093374357739825e18     16      24.150620097509638       35.92139018094681
50  1.0993264246156683e19    16      63.36958239742337        69.99768122728986
\end{verbatim}
\begin{verbatim}
Macierz losowa

n   c                        rank(A) error Gauss              error inv
5   1.000000000000001        5       1.4043333874306804e-16   2.2752801345137457e-16   
5   9.999999999999996        5       2.482534153247273e-16    2.432376777795247e-16
5   1000.0000000000236       5       2.9790409838967276e-16   4.203627514058621e-15
5   9.999999998130372e6      5       5.799015982916193e-11    1.377444663806729e-10
5   9.999322361505425e11     5       8.767946479620035e-6     1.101328684658636e-5
5   4.3389446864305475e15    4       0.5593130377167642       0.5591916598984644
10  1.000000000000001        10      5.15985034193911e-16     3.3121136700345433e-16
10  9.999999999999993        10      2.8737410463596867e-16   2.4575834280036907e-16
10  999.9999999999911        10      2.293995889930822e-14    2.7609708695270473e-14
10  9.999999990916912e6      10      2.0955792024649492e-10   1.81840723377465e-10
10  9.99976991136977e11      10      4.196755515487614e-5     3.614790502971321e-5
10  1.7803773341590864e16    9       0.14037179426228394      0.13931683203224224
20  1.0000000000000016       20      5.382005793715205e-16    3.9409007944299576e-16
20  10.0                     20      7.854386543748146e-16    6.309740391678007e-16
20  1000.0000000000028       20      3.120208680502288e-14    3.6192861929420846e-14   
20  9.99999999564815e6       20      1.603342362471366e-10    1.5697362145029217e-10
20  1.0000134366492958e12    20      1.6168699871775732e-5    1.4716047607150038e-5
20  8.563256185085127e15     19      0.3290975485091718       0.32066957992953976
\end{verbatim}

\subsection*{Wnioski}
\noindent Możemy zauważyć, że macierze Hilberta osiągają bardzo duże wartości współczynnika uwarunkowania (parametr \texttt{cond(A)}) już przy stosunkowo niewielkich rozmiarach n. Wysoki wskaźnik uwarunkowania oznacza, że układ $Ax = b$ jest źle uwarunkowany, czyli nawet niewielkie błędy w danych (w macierzy A lub w b) mogą prowadzić do dużych błędów w rozwiązaniu. Eksperymenty pokazują, że możemy to zaobserowować zarówno dla metody eliminacji Gauss'a jak i metody z użyciem odwrotności macierzy. Oznacza to, że numeryczne rozwiązanie układu Hilberta jest trudne do uzyskania z dużą dokładnością.\\
Dla losowych macierzy z ustalonym współczynnikiem uwarunkowania
c, błędy względne są małe i podobne dla obu metod. Oznacza to, że algorytmy są stabilne numerycznie dla macierzy dobrze uwarunkowanych.

\section{Zadanie 4}
\subsection*{Podpunkt a}
W zadaniu 4 mamy do czynienia z dwoma różnymi postaciami tego samego wielomianu. Mowa tutaj o wielomianie Wilkinsona. Jego postać iloczynowa wygląda tak:
\[
p(x) = (x - 20)\cdot(x - 19)\cdot(x - 18)\cdot...\cdot(x - 3)\cdot(x - 2)\cdot(x - 1)
\]
Z tej postaci od razu łatwo policzyć wszystkie miejsca zerowe:
\[
x_0 \in {1, 2, 3, ..., 18, 19, 20}
\]
Druga postać tego wielomianu to postać naturalna (współczynnikowa):
\begin{dmath}
P(x) = x^{20} - 210x^{19} + 20615x^{18} - 1256850x^{17} + 53327946x^{16}
-1672280820x^{15} + 40171771630x^{14} - 756111184500x^{13} + 11310276995381x^{12}
-135585182899530x^{11} + 1307535010540395x^{10} - 10142299865511450x^{9}
+63030812099294896x^{8} - 311333643161390640x^{7} + 1206647803780373360x^{6}
-3599979517947607200x^{5} + 8037811822645051776x^{4} - 12870931245150988800x^{3}
+13803759753640704000x^{2} - 8752948036761600000x + 2432902008176640000
\end{dmath}

\noindent Ta postać wynika bezpośrednio z wymnożenia wszystkich nawiasów z poprzedniej postaci. Liczenie miejsc zerowych jest w jej przypadku znacznie bardziej skomplikowane. Poniżej znajduje się porównanie rzeczywistych pierwiastków równania wraz z wyliczonymi pierwiastkami z wielomianu $P(x)$, wartością wielomianu p i P dla obliczonego miesjca zerowego, oraz różnicę między rzeczywistym a wyliczonym m. zer.
\noindent \\

\begin{tabular}{|c|c|c|c|c|}
\hline
k    & $z_k$                & $|P(z_k)|$           & $|p(z_k)|$           & $|z_k - k|$          \\\hline
1    & 0.9999999999998084   & 23323.616390897252   & 23310.180819556477   & 1.9162449405030202e-13 \\\hline
2    & 2.0000000000114264   & 64613.550791712885   & 73156.18130995684    & 1.1426415369442111e-11 \\\hline
3    & 2.9999999998168487   & 18851.098984644806   & 130289.20977428104   & 1.8315127192636282e-10 \\\hline
4    & 3.999999983818672    & 2.6359390809003003e6 & 2.0313511850413054e6 & 1.6181327833209025e-8 \\\hline
5    & 5.000000688670983    & 2.3709842874839526e7 & 2.1613359049100634e7 & 6.88670983350903e-7  \\\hline
6    & 5.999988371602095    & 1.2641076289358065e8 & 1.2165063267640364e8 & 1.162839790502801e-5 \\\hline
7    & 7.000112910766231    & 5.2301629899144447e8 & 5.061885949319773e8  & 0.00011291076623098917 \\\hline
8    & 7.999279406281878    & 1.798432141726085e9  & 1.7402728325721796e9 & 0.0007205937181220534 \\\hline
9    & 9.003273831140774    & 5.121881552672067e9  & 5.263758084354485e9  & 0.003273831140774064 \\\hline
10   & 9.989265687778465    & 1.4157542666785017e10 & 1.4147808356077827e10 & 0.010734312221535092 \\\hline
11   & 11.027997558569794   & 3.586354765112257e10 & 3.692632803664625e10 & 0.027997558569794023 \\\hline
12   & 11.94827395840048    & 8.510931555828575e10 & 8.162184753413098e10 & 0.051726041599520656 \\\hline
13   & 13.082031971969954   & 2.2136146301419052e11 & 2.04374354285492e11  & 0.08203197196995404  \\\hline
14   & 13.906800565193148   & 3.812024574451268e11 & 3.8519295444834576e11 & 0.09319943480685211  \\\hline
15   & 15.081439299377482   & 8.809029239560208e11 & 9.126025022282091e11 & 0.0814392993774824   \\\hline
16   & 15.942404318674466   & 1.6747434633806333e12 & 1.6751148968378867e12 & 0.05759568132553383  \\\hline
17   & 17.026861831476396   & 3.3067827086376123e12 & 3.5115331717998135e12 & 0.026861831476395537 \\\hline
18   & 17.99048462339055    & 6.166202940769282e12 & 6.644365795060319e12 & 0.009515376609449788 \\\hline
19   & 19.001981084996206   & 1.406783619602919e13 & 1.274643243051272e13 & 0.001981084996206306 \\\hline
20   & 19.999803908064397   & 3.284992217648231e13 & 2.3837033672895914e13 & 0.00019609193560299332 \\\hline

\end{tabular}

\subsubsection*{Wnioski}
Wyliczone miejsca zerowe są bardzo zbliżone do rzeczywistych miejsc zerowych. Jednak po podstawieniu ich z powrotem do wzoru otrzymujemy bardzo rozbieżne wyniki. Jak sama nazwa wskazuje miejsce zerowe to taki argument, dla którego funkcja osiąga wartość 0. Żadna z policzonych $P(z_k), p(z_k)$ nie jest nawet bliska 0. Otrzymujemy wartości od rzędu $10^4$ aż do $10^13$. Dzieje się tak, , ponieważ wielomian w postaci naturalnej przechowywany jest niedokładnie z uwagi na ograniczenia arytmetyki Float64. Niektóre współczynniki wielomianu przy niższych potęgach x są dużymi liczbami, których zapis pomija killka cyfr znaczących. Nie bez powodu wielomian jest nazwany złośliwym. Nawet bardzo małe oddalnie od precyzyjnej wartości miejsca zerowego, powoduje ogromne zaburzenia wyniku. Sugeruje to złe uwarunkowanie sposobu obliczania pierwiastków wielomianu.
 

\subsection*{Podpunkt b}
Celem eksperymentu Wilkinsona będącego tematem podpunktu b było zaburzenie jednego ze współczynników i obserwacja zachodzących zmian. Współczynnikiem, który uległ zmianie był $-210$. Teraz przyjmuje wartość $-210-2^{-23}$. 

\noindent Analogiczna tabela jak w przykładzie powyżej tym razem dla wielomianu o zaburzonym współczynniku.\\

\resizebox{\textwidth}{!}{%
\begin{tabular}{|c|c|c|c|c|}
\hline
k    & $z_k$                & $|P(z_k)|$           & $|p(z_k)|$           & $|z_k - k|$          \\\hline
1    & 0.9999999999999805 + 0.0im & 2168.9361669986724   & 2376.9361669985137   & 1.9539925233402755e-14 \\\hline
2    & 1.9999999999985736 + 0.0im & 29948.438957395843   & 9132.438957447212    & 1.4264145420384011e-12 \\\hline
3    & 3.000000000105087 + 0.0im & 239010.53520956426   & 74756.28518912957    & 1.0508705017286957e-10 \\\hline
4    & 3.9999999950066143 + 0.0im & 939293.8049425513    & 626853.3644463811    & 4.993385704921138e-9 \\\hline
5    & 5.000000034712704 + 0.0im & 7.44868039679552e6   & 1.0894298660834588e6 & 3.4712703822492585e-8 \\\hline
6    & 6.000005852511414 + 0.0im & 1.4689332508961653e7 & 6.12250863773296e7   & 5.852511414161654e-6 \\\hline
7    & 6.999704466216799 + 0.0im & 5.817946400915084e7  & 1.3252981745917735e9 & 0.00029553378320112955 \\\hline
8    & 8.007226654064777 + 0.0im & 1.3954205929609105e8 & 1.7380734718418133e10 & 0.0072266540647767386 \\\hline
9    & 8.917396943382494 + 0.0im & 2.459617755654851e8  & 1.3487291517349089e11 & 0.082603056617506    \\\hline
10   & 10.09529034477879 - 0.6432770896263527im & 2.291018560461982e9  & 1.4824347490765474e12 & 0.6502965968281023   \\\hline
11   & 10.09529034477879 + 0.6432770896263527im & 2.291018560461982e9  & 1.4824347490765474e12 & 1.110092326920887    \\\hline
12   & 11.793588728372308 - 1.6522535463872843im & 2.077690789102519e10 & 3.2939582416936758e13 & 1.6650968123818863   \\\hline
13   & 11.793588728372308 + 1.6522535463872843im & 2.077690789102519e10 & 3.2939582416936758e13 & 2.0458176697496047   \\\hline
14   & 13.99233053734825 - 2.5188196443048287im & 9.390730597798799e10 & 9.545412818924855e14 & 2.5188313205122075   \\\hline
15   & 13.99233053734825 + 2.5188196443048287im & 9.390730597798799e10 & 9.545412818924855e14 & 2.7129043747424584   \\\hline
16   & 16.73073008036981 - 2.8126272986972136im & 9.592356563898315e11 & 2.7420705762218132e16 & 2.906000476898456    \\\hline
17   & 16.73073008036981 + 2.8126272986972136im & 9.592356563898315e11 & 2.7420705762218132e16 & 2.8254873227453055   \\\hline
18   & 19.50243895868367 - 1.9403320231930836im & 5.050467401799687e12 & 4.2524547099295936e17 & 2.4540193937292005   \\\hline
19   & 19.50243895868367 + 1.9403320231930836im & 5.050467401799687e12 & 4.2524547099295936e17 & 2.004328632592893    \\\hline
20   & 20.84690887410499 + 0.0im & 4.858653129933677e12 & 1.374367214153433e18 & 0.8469088741049902   \\\hline
\end{tabular}
}

\subsubsection*{Wnioski}
Wykonany powyżej eksperyment potwierdza tezę, że zadanie jest źle uwarunkowane. Tym razem, niewielkie zaburzenie danych, powoduje pojawienie się pierwiastków zespolonych. Jest to coś kompletnie innego niż spodziewane wyniki. Wartości funkcji dla tych rzekomych pierwiastków są jeszcze większe niż w przypadku podpunktu a.

\section{Zadanie 5}
W tym zadaniu mamy rozważyć problem rekurencyjny dotyczący wzrotu populacji. Mamy równanie:
\[
p_{n+1} := p_n + r \cdot p_n \cdot (1 - p_n)
\]
określone dla n = {0, 1, 2, 3, ...} oraz r będącej pewną stałą.
Należało wykonać doświadczenie, w którym $p_0 = 0.01$ a $r = 3$.0. Liczba iteracji w każdym przypadku wynosiła 40, z tym, że rekurencję należało policzyć na dwa sposoby:
\begin{itemize}
    \item sposób 1 zakładał policzenie reukrencji klasycznie, w pętli 40-krotnie
    \item sposób drugi zakładał obcięcie wyniku do trzech cyfr po przecinku po wykonaniu 10 iteracji i nastęnie policzenie już klasycznie do końca.
\end{itemize}
Wyniki dla arytmetyki \texttt{Float32} zawarte są w dwóch pierwszych kolumnach. Pierwsza to sposób 1, druga sposób 2. W kolumnie trzeciej znajdują się wyniki dla sposobu 1 i arytmetyki \texttt{Float64}

\noindent
\begin{minipage}[t]{0.3\textwidth}
\begin{verbatim}
Iter | p
1    | 0.0397
2    | 0.15407173
3    | 0.5450726
4    | 1.2889781
5    | 0.1715188
6    | 0.5978191
7    | 1.3191134
8    | 0.056273222
9    | 0.21559286
10   | 0.7229306

11   | 1.3238364
12   | 0.037716985
13   | 0.14660022
14   | 0.521926
15   | 1.2704837
16   | 0.2395482
17   | 0.7860428
18   | 1.2905813
19   | 0.16552472
20   | 0.5799036
21   | 1.3107498
22   | 0.088804245
23   | 0.3315584
24   | 0.9964407
25   | 1.0070806
26   | 0.9856885
27   | 1.0280086
28   | 0.9416294
29   | 1.1065198
30   | 0.7529209
31   | 1.3110139
32   | 0.0877831
33   | 0.3280148
34   | 0.9892781
35   | 1.021099
36   | 0.95646656
37   | 1.0813814
38   | 0.81736827
39   | 1.2652004
40   | 0.25860548
\end{verbatim}
\end{minipage}%
\begin{minipage}[t]{0.3\textwidth}
\begin{verbatim}
Iter | p (modified)
1    | 0.0397
2    | 0.15407173
3    | 0.5450726
4    | 1.2889781
5    | 0.1715188
6    | 0.5978191
7    | 1.3191134
8    | 0.056273222
9    | 0.21559286
10   | 0.7229306
cut  | 0.722
11   | 1.3241479
12   | 0.036488414
13   | 0.14195944
14   | 0.50738037
15   | 1.2572169
16   | 0.28708452
17   | 0.9010855
18   | 1.1684768
19   | 0.577893
20   | 1.3096911
21   | 0.09289217
22   | 0.34568182
23   | 1.0242395
24   | 0.94975823
25   | 1.0929108
26   | 0.7882812
27   | 1.2889631
28   | 0.17157483
29   | 0.59798557
30   | 1.3191822
31   | 0.05600393
32   | 0.21460639
33   | 0.7202578
34   | 1.3247173
35   | 0.034241438
36   | 0.13344833
37   | 0.48036796
38   | 1.2292118
39   | 0.3839622
40   | 1.093568
\end{verbatim}
\end{minipage}%
\begin{minipage}[t]{0.3\textwidth}
\begin{verbatim}
Iter | p
1    | 0.0397
2    | 0.15407173000000002
3    | 0.5450726260444213
4    | 1.2889780011888006
5    | 0.17151914210917552
6    | 0.5978201201070994
7    | 1.3191137924137974
8    | 0.056271577646256565
9    | 0.21558683923263022
10   | 0.722914301179573

11   | 1.3238419441684408
12   | 0.03769529725473175
13   | 0.14651838271355924
14   | 0.521670621435246
15   | 1.2702617739350768
16   | 0.24035217277824272
17   | 0.7881011902353041
18   | 1.2890943027903075
19   | 0.17108484670194324
20   | 0.5965293124946907
21   | 1.3185755879825978
22   | 0.058377608259430724
23   | 0.22328659759944824
24   | 0.7435756763951792
25   | 1.315588346001072
26   | 0.07003529560277899
27   | 0.26542635452061003
28   | 0.8503519690601384
29   | 1.2321124623871897
30   | 0.37414648963928676
31   | 1.0766291714289444
32   | 0.8291255674004515
33   | 1.2541546500504441
34   | 0.29790694147232066
35   | 0.9253821285571046
36   | 1.1325322626697856
37   | 0.6822410727153098
38   | 1.3326056469620293
39   | 0.0029091569028512065
40   | 0.011611238029748606 
\end{verbatim}
\end{minipage}
\subsection*{Wnioski}
Możemy zauważyć, że wyniki dla sposobu 1 pokrywają się dość dokładnie w obydwu arytmatykach. Dopiero w iteracji 18 występuje różnica na 2 miejscu znaczącym , a w iteracji 23 na pierwszym. W okolicach 19 iteracji model na którym zastosowaliśmy sposób 2, czyli sposób z obcięciem, zaczyna mocno odstawać od wyników w pozostałych dwóch kolumnach. Jest to spowodowane faktem, że nasze rygorystyczne obcięcie (do trzeciego miejsca po przecinku) kumuluje się w kolejnych iteracjach. Program działa rekurencyjnie, a więc nasze uproszczenie zastosowane po kroku 10 nie daje o sobie zapomnieć w kolejnych iteracjach. Zwłaszcza, że we wzorze wartość $p_n$ jest podnoszona do kwadratu. Nasza utrata precyzji jest przez to jeszcze większa.

\noindent Wobec tego zauważamy, że zarówno zastosowane uproszczenie w sposobie 2 jak i stosowana arytmetyka mają znaczenie w precyzji obliczeń. Najbardziej godne zaufania i zapewne bliskie prawdy są obliczenia we \texttt{Float64} bez obcięć, jednak nawet ta aryemetyka i ten konkretny spsób nie gwarantują nam 100-procentowej dokładności.

\section{Zadanie 6}
W zadaniu 6 znów spotykamy się z rekurencją:
\[
x_{n+1} := {x_n}^2 + c
\]
dla n = {0, 1, 2, 3, ...}, gdzie c jest pewną daną stałą.
Eksperymenty mamy przeprowadzić dla danych:
\begin{itemize}
    \item $c = -2$ i $x_0 \in \{1, 2, 1.99999999999999\}$
    \item $c = -1$ i $x_0 \in \{1, -1, 0.75, 0.25\}$
\end{itemize}

\noindent Poniżej wartości otrzymane dla c = -2\\
\noindent
\begin{minipage}[t]{0.3\textwidth}
\begin{verbatim}
    x = 1.0
    Iter | x
    1    | -1.0
    2    | -1.0
    3    | -1.0
    4    | -1.0
    5    | -1.0
    6    | -1.0
    7    | -1.0
    8    | -1.0
    9    | -1.0
    10   | -1.0
    11   | -1.0
    12   | -1.0
    13   | -1.0
    14   | -1.0
    15   | -1.0
    16   | -1.0
    17   | -1.0
    18   | -1.0
    19   | -1.0
    20   | -1.0
    21   | -1.0
    22   | -1.0
    23   | -1.0
    24   | -1.0
    25   | -1.0
    26   | -1.0
    27   | -1.0
    28   | -1.0
    29   | -1.0
    30   | -1.0
    31   | -1.0
    32   | -1.0
    33   | -1.0
    34   | -1.0
    35   | -1.0
    36   | -1.0
    37   | -1.0
    38   | -1.0
    39   | -1.0
    40   | -1.0
\end{verbatim}
\end{minipage}%
\begin{minipage}[t]{0.3\textwidth}
\begin{verbatim}
    x = 2.0
    Iter | x
    1    | 2.0
    2    | 2.0
    3    | 2.0
    4    | 2.0
    5    | 2.0
    6    | 2.0
    7    | 2.0
    8    | 2.0
    9    | 2.0
    10   | 2.0
    11   | 2.0
    12   | 2.0
    13   | 2.0
    14   | 2.0
    15   | 2.0
    16   | 2.0
    17   | 2.0
    18   | 2.0
    19   | 2.0
    20   | 2.0
    21   | 2.0
    22   | 2.0
    23   | 2.0
    24   | 2.0
    25   | 2.0
    26   | 2.0
    27   | 2.0
    28   | 2.0
    29   | 2.0
    30   | 2.0
    31   | 2.0
    32   | 2.0
    33   | 2.0
    34   | 2.0
    35   | 2.0
    36   | 2.0
    37   | 2.0
    38   | 2.0
    39   | 2.0
    40   | 2.0
\end{verbatim}
\end{minipage}%
\begin{minipage}[t]{0.3\textwidth}
\begin{verbatim}
    x = 1.99999999999999
    Iter | x
    1    | 1.99999999999996
    2    | 1.9999999999998401
    3    | 1.9999999999993605
    4    | 1.999999999997442
    5    | 1.9999999999897682
    6    | 1.9999999999590727
    7    | 1.999999999836291
    8    | 1.9999999993451638
    9    | 1.9999999973806553
    10   | 1.999999989522621
    11   | 1.9999999580904841
    12   | 1.9999998323619383
    13   | 1.9999993294477814
    14   | 1.9999973177915749
    15   | 1.9999892711734937
    16   | 1.9999570848090826
    17   | 1.999828341078044
    18   | 1.9993133937789613
    19   | 1.9972540465439481
    20   | 1.9890237264361752
    21   | 1.9562153843260486
    22   | 1.82677862987391
    23   | 1.3371201625639997
    24   | -0.21210967086482313
    25   | -1.9550094875256163
    26   | 1.822062096315173
    27   | 1.319910282828443
    28   | -0.2578368452837396
    29   | -1.9335201612141288
    30   | 1.7385002138215109
    31   | 1.0223829934574389
    32   | -0.9547330146890065
    33   | -1.0884848706628412
    34   | -0.8152006863380978
    35   | -1.3354478409938944
    36   | -0.21657906398474625
    37   | -1.953093509043491
    38   | 1.8145742550678174
    39   | 1.2926797271549244
    40   | -0.3289791230026702
\end{verbatim}
\end{minipage}

\noindent \\\\Teraz wartości dla c = -1\\
\noindent
\begin{minipage}[t]{0.5\textwidth}
\begin{verbatim}
x = 1.0
Iter | x
1    | 0.0
2    | -1.0
3    | 0.0
4    | -1.0
5    | 0.0
6    | -1.0
7    | 0.0
8    | -1.0
9    | 0.0
10   | -1.0
11   | 0.0
12   | -1.0
13   | 0.0
14   | -1.0
15   | 0.0
16   | -1.0
17   | 0.0
18   | -1.0
19   | 0.0
20   | -1.0
21   | 0.0
22   | -1.0
23   | 0.0
24   | -1.0
25   | 0.0
26   | -1.0
27   | 0.0
28   | -1.0
29   | 0.0
30   | -1.0
31   | 0.0
32   | -1.0
33   | 0.0
34   | -1.0
35   | 0.0
36   | -1.0
37   | 0.0
38   | -1.0
39   | 0.0
40   | -1.0
\end{verbatim}
\end{minipage}%
\begin{minipage}[t]{0.5\textwidth}
\begin{verbatim}
x = -1.0
Iter | x
1    | 0.0
2    | -1.0
3    | 0.0
4    | -1.0
5    | 0.0
6    | -1.0
7    | 0.0
8    | -1.0
9    | 0.0
10   | -1.0
11   | 0.0
12   | -1.0
13   | 0.0
14   | -1.0
15   | 0.0
16   | -1.0
17   | 0.0
18   | -1.0
19   | 0.0
20   | -1.0
21   | 0.0
22   | -1.0
23   | 0.0
24   | -1.0
25   | 0.0
26   | -1.0
27   | 0.0
28   | -1.0
29   | 0.0
30   | -1.0
31   | 0.0
32   | -1.0
33   | 0.0
34   | -1.0
35   | 0.0
36   | -1.0
37   | 0.0
38   | -1.0
39   | 0.0
40   | -1.0
\end{verbatim}
\end{minipage}
\begin{minipage}[t]{0.5\textwidth}
\begin{verbatim}
x = 0.75
Iter | x
1    | -0.4375
2    | -0.80859375
3    | -0.3461761474609375
4    | -0.8801620749291033
5    | -0.2253147218564956
6    | -0.9492332761147301
7    | -0.0989561875164966
8    | -0.9902076729521999
9    | -0.01948876442658909
10   | -0.999620188061125
11   | -0.0007594796206411569
12   | -0.9999994231907058
13   | -1.1536182557003727e-6
14   | -0.9999999999986692
15   | -2.6616486792363503e-12
16   | -1.0
17   | 0.0
18   | -1.0
19   | 0.0
20   | -1.0
21   | 0.0
22   | -1.0
23   | 0.0
24   | -1.0
25   | 0.0
26   | -1.0
27   | 0.0
28   | -1.0
29   | 0.0
30   | -1.0
31   | 0.0
32   | -1.0
33   | 0.0
34   | -1.0
35   | 0.0
36   | -1.0
37   | 0.0
38   | -1.0
39   | 0.0
40   | -1.0
\end{verbatim}
\end{minipage}%
\begin{minipage}[t]{0.5\textwidth}
\begin{verbatim}
x = 0.25
Iter | x
1    | -0.9375
2    | -0.12109375
3    | -0.9853363037109375
4    | -0.029112368589267135
5    | -0.9991524699951226
6    | -0.0016943417026455965
7    | -0.9999971292061947
8    | -5.741579369278327e-6
9    | -0.9999999999670343
10   | -6.593148249578462e-11
11   | -1.0
12   | 0.0
13   | -1.0
14   | 0.0
15   | -1.0
16   | 0.0
17   | -1.0
18   | 0.0
19   | -1.0
20   | 0.0
21   | -1.0
22   | 0.0
23   | -1.0
24   | 0.0
25   | -1.0
26   | 0.0
27   | -1.0
28   | 0.0
29   | -1.0
30   | 0.0
31   | -1.0
32   | 0.0
33   | -1.0
34   | 0.0
35   | -1.0
36   | 0.0
37   | -1.0
38   | 0.0
39   | -1.0
40   | 0.0
\end{verbatim}
\end{minipage}




\end{document}