\documentclass{article}
\usepackage{subcaption}
\usepackage{graphicx}
\usepackage[T1]{fontenc}
\usepackage[utf8]{inputenc}
\usepackage[polish]{babel}
\usepackage{amsmath}
\usepackage{amsfonts}
\usepackage{float}

\setlength{\parindent}{0pt}

\title{Sprawozdanie laboratorium lista 5 \\Obliczenia naukowe}
\author{\normalsize Zofia Tarchalska, indeks: 279699}
\date{}

\begin{document}
\maketitle
\section{Wstęp}
Celem tej listy było zaimplementowanie trzech poniższych algorytmów o złożoności $O(n)$:
\begin{itemize}
    \item funkcja rozwiązująca układ $Ax=b$ metodą eliminacji Gaussa
    \item funkcja wyznaczająca rozkład $LU$ maciezry $A$ metodą eliminacji Gaussa
    \item funkcja rozwiązująca układ $Ax=b$ jeśli już wcześniej został wyznaczony rozkład $LU$
\end{itemize}
Wszystkie te algorytmy mają działać dla macierzy $A$ o specyficznej postaci, która została dokładnie opisana w poleceniu do zadania.

\section{Metoda eliminacji Gaussa}
\subsection{Wariant pierwszy - bez wyboru elementu głównego}
Główną ideą metody eliminacji Gaussa jest doprowadzenie macierzy do postaci, w której pod przekątną znajdują się same zera. Uzyskujemy to poprzez mnożenie kolejnych wierszy macierzy przez odpowiednie czynniki i odejmowanie ich od wierszy następujących po nich. Poprawny schemat:
\newline

Eliminujemy zmienną $x_k$ z wierszy od $k + 1$ do $n$. Mnożymy $k-te$ równanie przez 
\[
    l_{ik} = \frac{a_{ik}}{a_{kk}} \quad \text{dla } i \in \{k + 1, ..., n\}
\]

Kiedy jednak na przekątnej w miejscu $a_{kk}$ będzie $0$ metoda może powodować błąd numeryczny, ponieważ nastąpi dzielenie przez $0$. W przeciwnym wypadku, po wykonaniu odpowiednich kroków kolejno dla wszystkich wierszy macierzy, otrzymujemy macierz górnotrójkątną. 




TO DO opis algorytmu 


\subsection{Wariant drugi - z częściowym wyborem \\elementu głównego}
Ten wariant ma zabezpieczać nas przed potencjalnym dzieleniem przez $0$ gdy na przekątnej macierzy taka warotść się znajduje. Biorąc pod uwagę, że obliczenia wykonujemy na komputerze, liczby bardzo zbliżone do $0$ również są tymi, które mogą nam zwrócić błąd.

Element główny to nic innego jak wybrana wartość, którą będziemy używać aby wyzerować pozostałe w kolumnie. 




\end{document}